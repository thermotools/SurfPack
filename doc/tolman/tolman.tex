\makeatletter
\def\input@path{{../dftmemo/}}
\makeatother
\documentclass[english]{dftmemo}
\pdfminorversion=4
\usepackage[utf8]{inputenc}
\usepackage{textcomp}
\usepackage{listings}
\usepackage{babel}
\usepackage[square,numbers]{natbib}
%\usepackage{amsmath}
\usepackage{amssymb,amsmath,amsthm,amsfonts}
\usepackage{pslatex}
%\usepackage{bm}% better \boldsymbol
\usepackage{array}% improves tabular environment.
\usepackage{dcolumn}% also improves tabular environment, with decimal centring.
%\usepackage{lastpage}
\usepackage{listings}
\usepackage{tikz}
\usetikzlibrary{decorations.pathmorphing}
%\usepackage{pgfplots}
%\usepackage{cleveref}
\usepackage{pgf,pgfarrows,pgfnodes}

\usepackage[final]{pdfpages}
\usepackage{hyperref}
\usepackage{doi}

%\usepackage[altbullet,expert]{lucidabr}
%\usepackage[charter]{mathdesign}
\usepackage{cleveref}
\usepackage{ifthen}
%\usepackage{nomencl}
\usepackage{parskip}
\usepackage{psfrag}
\usepackage{booktabs}
\usepackage{pifont}
%\usepackage{subfig,caption}
\usepackage{microtype}% for pdflatex; medisin mot overfull \vbox
\usepackage{xspace}% for pdflatex; medisin mot overfull \vbox
\usepackage[font=footnotesize]{subcaption}
\usepackage{siunitx}
\usepackage{todonotes}
\presetkeys%
    {todonotes}%
    {inline,backgroundcolor=orange}{}

%\newcommand*{\od}[3][]{\frac{\mathrm{d}^{#1}#2}{\mathrm{d}{#3}^{#1}}}% ordinary derivative
\newcommand*{\od}[3][]{\frac{\dif^{#1}#2}{\dif{#3}^{#1}}}% ordinary derivative
\newcommand*{\pd}[3][]{\frac{\partial^{#1}#2}{\partial{#3}^{#1}}}% partial derivative
\newcommand*{\pdt}[3][]{{\partial^{#1}#2}/{\partial{#3}^{#1}}}% partial
                                % derivative for inline use.
\newcommand{\pone}[3]{\frac{\partial #1}{\partial #2}_{#3}}% partial
                                % derivative with information of
                                % constant variables
\newcommand{\ponel}[3]{\frac{\partial #1}{\partial #2}\bigg|_{#3}} % partial derivative with informatio of constant variable. A line is added.
\newcommand{\ptwo}[3]{\frac{\partial^{2} #1}{\partial #2 \partial
    #3}} % partial differential in two different variables
\newcommand{\pdn}[3]{\frac{\partial^{#1}#2}{\partial{#3}^{#1}}}% partial derivative

% Total derivative:
\newcommand*{\ttd}[2]{\frac{\mathrm{D} #1}{\mathrm{D} #2}}
\newcommand*{\td}[2]{\frac{\mathrm{d} #1}{\mathrm{d} #2}}
\newcommand*{\ddt}{\frac{\partial}{\partial t}}
\newcommand*{\ddx}{\frac{\partial}{\partial x}}
\newcommand*{\sgn}[1]{\text{sgn}(#1)}
% Vectors etc:
% For Computer Modern:

\DeclareMathAlphabet{\mathsfsl}{OT1}{cmss}{m}{sl}
%\renewcommand*{\vec}[1]{\boldsymbol{#1}}%
\newcommand*{\vc}[1]{\vec{\mathbf{#1}}}%
\newcommand*{\tensor}[1]{\mathsfsl{#1}}% 2. order tensor
\newcommand*{\matr}[1]{\tensor{#1}}% matrix
\renewcommand*{\div}{\boldsymbol{\nabla\cdot}}% divergence
\newcommand*{\grad}{\boldsymbol{\nabla}}% gradient
% fancy differential from Claudio Beccari, TUGboat:
% adjusts spacing automatically
\makeatletter
\newcommand*{\dif}{\@ifnextchar^{\DIfF}{\DIfF^{}}}
\def\DIfF^#1{\mathop{\mathrm{\mathstrut d}}\nolimits^{#1}\gobblesp@ce}
\def\gobblesp@ce{\futurelet\diffarg\opsp@ce}
\def\opsp@ce{%
  \let\DiffSpace\!%
  \ifx\diffarg(%
    \let\DiffSpace\relax
  \else
    \ifx\diffarg[%
      \let\DiffSpace\relax
    \else
      \ifx\diffarg\{%
        \let\DiffSpace\relax
      \fi\fi\fi\DiffSpace}
\makeatother
%
\newcommand*{\abs}[1]{\lvert#1\rvert}
\newcommand*{\bigabs}[1]{\bigl\lvert#1\bigr\rvert}
\newcommand*{\biggabs}[1]{\biggl\lvert#1\biggr\rvert}
\newcommand*{\norm}[1]{\lVert#1\rVert}
\newcommand{\subl}{\ensuremath{\text{sub}}}
\newcommand{\spec}{\text{spec}}
\newcommand{\sat}{\text{sat}}
\newcommand{\sol}{\text{sol}}
\newcommand{\liq}{\text{liq}}
\newcommand{\vap}{\text{vap}}
\newcommand{\amb}{\text{amb}}
\newcommand{\tr}{\text{tr}}
\newcommand{\crit}{\text{crit}}
\newcommand{\entr}{\ensuremath{\text{s}}}
\newcommand{\fus}{\text{fus}}
\newcommand*{\lb}{\ensuremath{\left(}}
\newcommand*{\rb}{\ensuremath{\right)}}
\newcommand*{\lbf}{\ensuremath{\left[}}
\newcommand*{\rbf}{\ensuremath{\right]}}
\newcommand{\LEFT}{\ensuremath{\text{L}}\xspace}
\newcommand{\RIGTH}{\ensuremath{\text{R}}\xspace}
\newcommand{\cdft}{\ensuremath{\text{classical DFT}}\xspace}
\newcommand{\RF}{\ensuremath{\text{RF}}\xspace}
\newcommand{\WB}{\ensuremath{\text{WB}}\xspace}
\newcommand{\WBII}{\ensuremath{\text{WBII}}\xspace}
\newcommand{\excess}{\ensuremath{\text{ex}}\xspace}
\newcommand{\ideal}{\ensuremath{\text{id}}\xspace}
\newcommand{\rvec}{\ensuremath{\mathbf{r}}\xspace}
\newcommand{\attractive}{\ensuremath{\text{att}}\xspace}
\newcommand{\kB}{\ensuremath{\text{k}_{\text{B}}}\xspace}
\newcommand{\NA}{\ensuremath{N_{\text{A}}}\xspace}
\newcommand{\external}{\ensuremath{\text{ext}}\xspace}
\newcommand{\bulk}{\ensuremath{\text{b}}\xspace}
\newcommand{\pure}{\ensuremath{\text{p}}\xspace}
\newcommand{\NC}{\ensuremath{\text{NC}}\xspace}
\newcommand{\disp}{\ensuremath{\text{disp}}\xspace}
\graphicspath{{gfx/}}

\title{Tolman length and rigidities from DFT}
\author{Morten Hammer}
\date{\today}

\begin{document}
\frontmatter

\section{Introduction}

\citet{rehner2019} derived the equations needed to calculate the
Tolman length and rigidity constants used when expanding the planar
surface tension in the curvature of a droplet/bubble. This memo
describes the implementation in \emph{thermopack\_dft}.


\section{Fourier transforms}
The required convolution weight functions is tabulated in
\cite[Tab. II]{rehner2019}, and require differentials of the regular
weight functions, $\omega$.

Scalar weight functions:
\begin{equation}
  \mathcal{F}\lb z \omega \rb = \frac{i}{2 \pi} \omega^\prime \lb k \rb
\end{equation}

\begin{equation}
  \mathcal{F}\lb \tilde{\omega} \rb = \frac{i}{4 \pi^2} \biggl( \frac{\omega^\prime \lb k \rb}{k} - \omega^{\prime\prime} \lb k \rb \biggr)
\end{equation}

Vector weight functions:
\begin{equation}
  \mathcal{F}\lb z \omega_z \pm \hat{\omega}_z \rb = \frac{i}{2 \pi} \biggl( \omega_z^\prime \lb k \rb \pm \frac{\omega_z \lb k \rb}{k} \biggr)\biggr)
\end{equation}

\begin{equation}
  \mathcal{F}\lb z^2 \omega_z \pm z\hat{\omega}_z \rb = \frac{-1}{4 \pi^4} \biggl( \omega_z^{\prime\prime} \lb k \rb \pm \biggl( \frac{\omega_z^\prime \lb k \rb}{k} - \frac{\omega_z \lb k \rb}{k^2}\biggr)\biggr)
\end{equation}

Differentials, in Fourier space, is required for the weight functions.

The weight functions in Fourier space is given in the supplementary information of \citet{stierle2020a}.
\begin{align}
  \label{eq:weight_functions_fourier}
  \omega_0 \lb \mathbf{k}\rb &= j_0\lb 2 \pi R |\mathbf{k}| \rb  \\
  \omega_1 \lb \mathbf{k}\rb &= R j_0\lb 2 \pi R |\mathbf{k}| \rb  \\
  \omega_2 \lb \mathbf{k}\rb &= 4 \pi R^2 j_0\lb 2 \pi R |\mathbf{k}| \rb  \\
  \omega_3 \lb \mathbf{k}\rb &= \frac{4}{3} \pi R^3 \lb j_0\lb 2 \pi R |\mathbf{k}| \rb + j_2\lb 2 \pi R |\mathbf{k}| \rb \rb  \\
  \omega_{z1} \lb \mathbf{k}\rb &= - \frac{i \mathbf{k}}{2 R} \omega_3 \lb \mathbf{k}\rb  \\
  \omega_{z2} \lb \mathbf{k}\rb &= - 2 \pi i \mathbf{k} \omega_3 \lb \mathbf{k}\rb  \\
  \omega_{\disp} \lb \mathbf{k}\rb &= j_0\lb 4 \pi \psi R |\mathbf{k}| \rb + j_2\lb 4 \pi \psi R |\mathbf{k}| \rb
\end{align}
Here $j$ is the spherical Bessel functions of the first kind of order
zero and two. The Bessel functions of order 0-2 is given as
\begin{align}
  \label{eq:bessel_functions}
  j_0\lb \xi \rb &= \frac{\sin \lb \xi \rb}{\xi} \\
  j_1\lb \xi \rb &= \frac{\sin \lb \xi \rb}{\xi^2} - \frac{\cos \lb \xi \rb}{\xi}  \\
  j_2\lb \xi \rb &= \biggl(\frac{3}{\xi^2} - 1\biggr) \frac{\sin \lb \xi \rb}{\xi} - \frac{3 \cos \lb \xi \rb}{\xi^2}.
\end{align}

Using $k_* = 2 \pi k$, and differentiating with respect to $k_*$ we get,
\begin{align}
  \mathcal{F}\lb z \omega \rb &= i \omega^\prime \lb k_* \rb \\
  \mathcal{F}\lb \tilde{\omega} \rb &= i\biggl( \frac{\omega^\prime \lb k_* \rb}{k_*} - \omega^{\prime\prime} \lb k_* \rb \biggr) \\
  \mathcal{F}\lb z \omega_z \pm \hat{\omega}_z \rb &= i \biggl( \omega_z^\prime \lb k_* \rb \pm \frac{\omega_z \lb k_* \rb}{k_*} \biggr)\biggr) \\
  \mathcal{F}\lb z^2 \omega_z \pm z\hat{\omega}_z \rb &= -\biggl( \omega_z^{\prime\prime} \lb k_* \rb \pm \biggl( \frac{\omega_z^\prime \lb k_* \rb}{k_*} - \frac{\omega_z \lb k_* \rb}{k_*^{2}}\biggr)\biggr)
\end{align}

\subsection{Differentials}
The Bessel functions are included in SymPy, and the differentials can
be handled analytically. Also, as we are differentiating with respect
to $k$, it is not necessary to convolve with all the weights of the
Fundamental Measure Theory (FMT), as the differentials of
$\omega_0, \omega_1$ and $\omega_2$ are related by constant scalar
values. The same applies for $\omega_{z1}$ and $\omega_{z2}$.

The use of analytical differentials where very slow, and was not feasible.

Differentiating the required spherical Bessel functions gives after some manipulation,
\begin{align}
  \label{eq:bessel_function_differentials}
  \pd{j_0\lb \xi \rb}{\xi} &= -\frac{\sin \lb \xi \rb}{\xi^2} + \frac{\cos \lb \xi \rb}{\xi} = - j_1\lb \xi \rb\\
  \pd[2]{j_0\lb \xi \rb}{\xi} &= -\pd{}{\xi} \biggl(\frac{\sin \lb \xi \rb}{\xi^2} - \frac{\cos \lb \xi \rb}{\xi} \biggr) \nonumber \\
                           &= \frac{2 \sin \lb \xi \rb }{\xi^3} -  \frac{2 \cos \lb \xi \rb }{\xi^2} - \frac{\sin \lb \xi \rb }{\xi} \nonumber \\
                           &= \frac{1}{3} \biggl( 2 j_2\lb \xi \rb - j_0\lb \xi \rb \biggr) \\
  \pd{\left[\lb j_0\lb \xi \rb + j_2\lb \xi \rb\rb\right]}{\xi} &= \pd{}{\xi} \biggl(\frac{3\sin \lb \xi \rb}{\xi^3} - \frac{3\cos \lb \xi \rb}{\xi^2}\biggr) \nonumber \\
                           &= \frac{-9 \sin \lb \xi \rb }{\xi^4} +  \frac{9 \cos \lb \xi \rb }{\xi^3} + \frac{3 \sin \lb \xi \rb }{\xi^2} \nonumber \\
                           &= -\frac{9}{15} \biggl( j_3\lb \xi \rb + j_1\lb \xi \rb \biggr) \\
  \pd[2]{\left[\lb j_0\lb \xi \rb + j_2\lb \xi \rb\rb\right]}{\xi} &= \frac{36 \sin \lb \xi \rb }{\xi^5} -  \frac{36 \cos \lb \xi \rb }{\xi^4} - \frac{15 \sin \lb \xi \rb }{\xi^3} + \frac{3 \cos \lb \xi \rb }{\xi^2} \nonumber \\
                           &= \frac{1}{105} \biggl( 36 j_4\lb \xi \rb + 15 j_2\lb \xi \rb - 21 j_0\lb \xi \rb \biggr) \\
\end{align}

Using the relation
\begin{equation}
  \od{\xi}{k^*} = \od{\xi}{k^*} \lb R^* \abs{k^*} \rb= \sgn{k^*}R^*,
\end{equation}
the needed differentials are easy to write out.

\section{Results}



\clearpage
\bibliographystyle{abbrvnat}
\bibliography{../bib/DFT.bib,../bib/HardSphere.bib,../bib/PC-SAFT.bib}

\end{document}

%%% Local Variables: 
%%% mode: latex
%%% TeX-master: t
%%% ispell-local-dictionary: "british"
%%% End: 
