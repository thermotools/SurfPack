\documentclass[12pt, letterpaper]{article}
\pdfminorversion=4
\usepackage[utf8]{inputenc}
\usepackage{textcomp}
\usepackage{listings}
\usepackage{babel}
\usepackage[square,numbers]{natbib}
\usepackage{amsmath}
\usepackage{pslatex}
%\usepackage{bm}% better \boldsymbol
\usepackage{array}% improves tabular environment.
\usepackage{dcolumn}% also improves tabular environment, with decimal centring.
%\usepackage{lastpage}
\usepackage{listings}
\usepackage{tikz}
\usetikzlibrary{decorations.pathmorphing}
%\usepackage{pgfplots}
%\usepackage{cleveref}
\usepackage{pgf,pgfarrows,pgfnodes}

\usepackage[final]{pdfpages}
\usepackage{hyperref}
\usepackage{doi}

%\usepackage[altbullet,expert]{lucidabr}
%\usepackage[charter]{mathdesign}
\usepackage{cleveref}
\usepackage{ifthen}
%\usepackage{nomencl}
\usepackage{parskip}
\usepackage{psfrag}
\usepackage{booktabs}
\usepackage{pifont}
%\usepackage{subfig,caption}
\usepackage{microtype}% for pdflatex; medisin mot overfull \vbox
\usepackage{xspace}% for pdflatex; medisin mot overfull \vbox
\usepackage[font=footnotesize]{subcaption}
\usepackage{siunitx}
\usepackage{todonotes}
\presetkeys%
    {todonotes}%
    {inline,backgroundcolor=orange}{}

%% Formattering av figur- og tabelltekster
%
%
% Egendefinerte
%
\newcommand*{\unit}[1]{\ensuremath{\,\mathrm{#1}}}
\newcommand*{\uunit}[1]{\ensuremath{\mathrm{#1}}}
%\newcommand*{\od}[3][]{\frac{\mathrm{d}^{#1}#2}{\mathrm{d}{#3}^{#1}}}% ordinary derivative
\newcommand*{\od}[3][]{\frac{\dif^{#1}#2}{\dif{#3}^{#1}}}% ordinary derivative
\newcommand*{\pd}[3][]{\frac{\partial^{#1}#2}{\partial{#3}^{#1}}}% partial derivative
\newcommand*{\pdt}[3][]{{\partial^{#1}#2}/{\partial{#3}^{#1}}}% partial
                                % derivative for inline use.
\newcommand{\pone}[3]{\frac{\partial #1}{\partial #2}_{#3}}% partial
                                % derivative with information of
                                % constant variables
\newcommand{\ponel}[3]{\frac{\partial #1}{\partial #2}\bigg|_{#3}} % partial derivative with informatio of constant variable. A line is added.
\newcommand{\ptwo}[3]{\frac{\partial^{2} #1}{\partial #2 \partial
    #3}} % partial differential in two different variables
\newcommand{\pdn}[3]{\frac{\partial^{#1}#2}{\partial{#3}^{#1}}}% partial derivative

% Total derivative:
\newcommand*{\ttd}[2]{\frac{\mathrm{D} #1}{\mathrm{D} #2}}
\newcommand*{\td}[2]{\frac{\mathrm{d} #1}{\mathrm{d} #2}}
\newcommand*{\ddt}{\frac{\partial}{\partial t}}
\newcommand*{\ddx}{\frac{\partial}{\partial x}}
% Vectors etc:
% For Computer Modern:

\DeclareMathAlphabet{\mathsfsl}{OT1}{cmss}{m}{sl}
%\renewcommand*{\vec}[1]{\boldsymbol{#1}}%
\newcommand*{\vc}[1]{\vec{\mathbf{#1}}}%
\newcommand*{\tensor}[1]{\mathsfsl{#1}}% 2. order tensor
\newcommand*{\matr}[1]{\tensor{#1}}% matrix
\renewcommand*{\div}{\boldsymbol{\nabla\cdot}}% divergence
\newcommand*{\grad}{\boldsymbol{\nabla}}% gradient
% fancy differential from Claudio Beccari, TUGboat:
% adjusts spacing automatically
\makeatletter
\newcommand*{\dif}{\@ifnextchar^{\DIfF}{\DIfF^{}}}
\def\DIfF^#1{\mathop{\mathrm{\mathstrut d}}\nolimits^{#1}\gobblesp@ce}
\def\gobblesp@ce{\futurelet\diffarg\opsp@ce}
\def\opsp@ce{%
  \let\DiffSpace\!%
  \ifx\diffarg(%
    \let\DiffSpace\relax
  \else
    \ifx\diffarg[%
      \let\DiffSpace\relax
    \else
      \ifx\diffarg\{%
        \let\DiffSpace\relax
      \fi\fi\fi\DiffSpace}
\makeatother
%
\newcommand*{\me}{\mathrm{e}}% e is not a variable (2.718281828...)
%\newcommand*{\mi}{\mathrm{i}}%  nor i (\sqrt{-1})
\newcommand*{\mpi}{\uppi}% nor pi (3.141592...) (works for for Lucida)
%
% lav tekst-indeks/subscript/pedex
\newcommand*{\ped}[1]{\ensuremath{_{\text{#1}}}}
% hy tekst-indeks/superscript/apex
\newcommand*{\ap}[1]{\ensuremath{^{\text{#1}}}}
\newcommand*{\apr}[1]{\ensuremath{^{\mathrm{#1}}}}
\newcommand*{\pedr}[1]{\ensuremath{_{\mathrm{#1}}}}
%
\newcommand*{\volfrac}{\alpha}% volume fraction
\newcommand*{\surften}{\sigma}% coeff. of surface tension
\newcommand*{\curv}{\kappa}% curvature
\newcommand*{\ls}{\phi}% level-set function
\newcommand*{\ep}{\Phi}% electric potential
\newcommand*{\perm}{\varepsilon}% electric permittivity
\newcommand*{\visc}{\mu}% molecular (dymamic) viscosity
\newcommand*{\kvisc}{\nu}% kinematic viscosity
\newcommand*{\cfl}{C}% CFL number

% Grid
\newcommand{\jj}{j}
\newcommand{\jph}{{j+1/2}}
\newcommand{\jmh}{{j-1/2}}
\newcommand{\jp}{{j+1}}
\newcommand{\jm}{{j-1}}
\newcommand{\nn}{n}
\newcommand{\nph}{{n+1/2}}
\newcommand{\nmh}{{n-1/2}}
\newcommand{\np}{{n+1}}
%
\newcommand{\lf}{\text{LF}}
\newcommand{\lw}{\text{Ri}}
%
\newcommand*{\cons}{\vec U}
\newcommand*{\flux}{\vec F}
\newcommand*{\dens}{\rho}
\newcommand*{\svol}{\ensuremath v}
\newcommand*{\temp}{\ensuremath T}
\newcommand*{\vel}{\ensuremath u}
\newcommand*{\mom}{\dens\vel}
\newcommand*{\toten}{\ensuremath E}
\newcommand*{\inten}{\ensuremath e}
\newcommand*{\press}{\ensuremath p}
\renewcommand*{\ss}{\ensuremath a}
\newcommand*{\jac}{\matr A}
%
\newcommand*{\abs}[1]{\lvert#1\rvert}
\newcommand*{\bigabs}[1]{\bigl\lvert#1\bigr\rvert}
\newcommand*{\biggabs}[1]{\biggl\lvert#1\biggr\rvert}
\newcommand*{\norm}[1]{\lVert#1\rVert}
%
\newcommand*{\e}[1]{\times 10^{#1}}
\newcommand*{\ex}[1]{\times 10^{#1}}%shorthand -- for use e.g. in tables
\newcommand*{\exi}[1]{10^{#1}}%shorthand -- for use e.g. in tables
\newcommand*{\nondim}[1]{\ensuremath{\mathit{#1}}}% italic iflg. ISO. (???)
\newcommand*{\rey}{\nondim{Re}}
\newcommand*{\acro}[1]{\textsc{\MakeLowercase{#1}}}%acronyms etc.

\newcommand{\nto}{\ensuremath{\mbox{N}_{\mbox{\scriptsize 2}}}}
\newcommand{\chfire}{\ensuremath{\mbox{CH}_{\mbox{\scriptsize 4}}}}
%\newcommand*{\checked}{\ding{51}}
\newcommand{\coto}{\ensuremath{\text{CO}_{\text{\scriptsize 2}}}}
\newcommand{\celsius}{\ensuremath{^\circ\text{C}}}
%\newcommand{\clap}{Clapeyron~}
\newcommand{\subl}{\ensuremath{\text{sub}}}
\newcommand{\spec}{\text{spec}}
\newcommand{\sat}{\text{sat}}
\newcommand{\sol}{\text{sol}}
\newcommand{\liq}{\text{liq}}
\newcommand{\vap}{\text{vap}}
\newcommand{\amb}{\text{amb}}
\newcommand{\tr}{\text{tr}}
\newcommand{\crit}{\text{crit}}
\newcommand{\entr}{\ensuremath{\text{s}}}
\newcommand{\fus}{\text{fus}}
\newcommand{\flash}[1]{\ensuremath{#1\text{-flash}}}
\newcommand{\spce}[2]{\ensuremath{#1\, #2\text{ space}}}
\newcommand{\spanwagner}{\text{Span--Wagner}}
\newcommand{\triplepoint}{\text{TP triple point}}
\newcommand{\wrpt}{\text{with respec to~}}
%\sisetup{input-symbols = {( )}}
% \newcommand*{\red}{\ensuremath{\text{R}}\xspace}
% \newcommand*{\crit}{\ensuremath{\text{c}}\xspace}
% \newcommand*{\mix}{\ensuremath{\text{m}}\xspace}
\newcommand*{\lb}{\ensuremath{\left(}}
\newcommand*{\rb}{\ensuremath{\right)}}
\newcommand{\LEFT}{\ensuremath{\text{L}}\xspace}
\newcommand{\RIGTH}{\ensuremath{\text{R}}\xspace}
\newcommand{\cdft}{\ensuremath{\text{classical DFT}}\xspace}
\newcommand{\RF}{\ensuremath{\text{RF}}\xspace}
\newcommand{\WB}{\ensuremath{\text{WB}}\xspace}
\newcommand{\WBII}{\ensuremath{\text{WBII}}\xspace}

\graphicspath{{gfx/}}

\title{Classical Density Functional Theory (cDFT) for Thermopack}
\author{Morten Hammer}
\date{\today}

\begin{document}
%\tableofcontents
%\printnomenclature


\begin{titlepage}
\maketitle
\end{titlepage}

\section{Introduction}

The Jupiter notebooks of Mary K. Coe
\href{https://github.com/marykcoe/cDFT_Package}{cDFT} is a great recourse for
understanding \cdft. Her PhD thesis also contains a lot of information
\cite{coe2021}.

\section{Fundamental Measure Theory}
Fundamental measure theory for hard sphere mixtures was developed by
\citet{rosenfeld1989}. The name "measure" relates to the fundamental
geometrical measures (volume, surface area, mean radius of curvature
and the Euler characteristic) of a sphere particle. The fundamental
geometrical measures are recovered when integrating the weight
functions defined in Section~\ref{sec:weight}.

For bulk phases this functional reduces to the Percus-Yevick (PY)
compressibility equation \cite{percus1958}, equivalent to scaled
particle theory.

\subsection{The Rosenfeld functional}

\begin{equation}
  \Phi^\RF = -n_0 \ln \lb 1 - n_3 \rb +
  \frac{n_1 n_2 - \vc{n}_1 \cdot \vc{n}_2}{1 - n_3} +
  \frac{n_2^3 - 3 n_2 \vc{n}_2 \cdot \vc{n}_2}{24\pi \lb 1 - n_3 \rb^2}
\end{equation}

The differentials needed when searching for the Grand potential:
\begin{align}
  \pd{\Phi^\RF}{n_0} &= -\ln \lb 1 - n_3 \rb \\
  \pd{\Phi^\RF}{n_1} &= \frac{n_2}{1 - n_3} \\
  \pd{\Phi^\RF}{n_2} &= \frac{n_1}{1 - n_3} + \frac{n_2^2 - \vc{n}_2 \cdot \vc{n}_2}{8\pi \lb 1 - n_3 \rb^2} \\
  \pd{\Phi^\RF}{n_3} &= \frac{n_0}{1 - n_3} +
  \frac{n_1 n_2 - \vc{n}_1 \cdot \vc{n}_2}{\lb 1 - n_3 \rb^2} +
  \frac{n_2^3 - 3 n_2 \vc{n}_2 \cdot \vc{n}_2}{12\pi \lb 1 - n_3 \rb^3} \\
  \pd{\Phi^\RF}{\vc{n}_1} &=  - \frac{\vc{n}_2}{1 - n_3} \\
  \pd{\Phi^\RF}{\vc{n}_2} &=  -\frac{\vc{n}_1}{1 - n_3} - \frac{n_2 \vc{n}_2}{4\pi \lb 1 - n_3 \rb^2}
\end{align}

\subsection{Weight functions}
\label{sec:weight}
Weight functions given by

\begin{align}
  w_3^i \lb \mathbf{r} \rb &=  \Theta \lb R_i - \abs{\mathbf{r}} \rb \\
  w_2^i \lb \mathbf{r} \rb &=  \delta \lb R_i - \abs{\mathbf{r}} \rb \\
  w_1^i \lb \mathbf{r} \rb &= \frac{1}{4 \pi R_i} w_2^i \lb \mathbf{r} \rb \\
  w_0^i \lb \mathbf{r} \rb &= \frac{1}{4 \pi R_i^2} w_2^i \lb \mathbf{r} \rb \\
  \mathbf{w}_2^i \lb \mathbf{r} \rb &=  \frac{\mathbf{r}}{\abs{\mathbf{r}}}\delta \lb R_i - \abs{\mathbf{r}} \rb \\
  \mathbf{w}_1^i \lb \mathbf{r} \rb &= \frac{1}{4 \pi R_i} \mathbf{w}_2^i.
\end{align}
Where $\Theta$ is the Heaviside function, and $\delta$ are the Dirac delta function.

\subsubsection{Weight functions for planar geometry}
For the planar geometry $\rho \lb \mathbf{r} \rb = \rho \lb z \rb $,
and the weigth functions can be integrated for the $x,y$ dimensions.

\begin{equation}
  W_v \lb z \rb = \underset{-\infty}{\overset{\infty}{\int}} \underset{-\infty}{\overset{\infty}{\int}} dx dy w_v \lb \sqrt{x^2 + y^2 + z^2} \rb = 2 \pi \underset{\abs{z}}{\overset{\infty}{\int}} dr r  w_v \lb r \rb
\end{equation}

This can be integrated analytically to
\begin{align}
  w_3^i \lb z \rb &=  \pi \lb R_i^2 - z^2 \rb \Theta \lb R_i - \abs{z} \rb \\
  w_2^i \lb z \rb &=  2 \pi R_i \Theta \lb R_i - \abs{z} \rb \\
  \mathbf{w}_2^i \lb z \rb &= 2 \pi z \mathbf{e}_z  \Theta \lb R_i - \abs{z} \rb
\end{align}

\subsubsection{Weight functions for spherical geometry}

For the shperical geometry $\rho \lb \mathbf{r} \rb = \rho \lb r \rb $,
and the weigth functions can be integrated for the angle dimensions.

\begin{equation}
  W_v \lb r \rb = \underset{-\infty}{\overset{\infty}{\int}} \underset{-\infty}{\overset{\infty}{\int}} dx dy w_v \lb \sqrt{x^2 + y^2 + z^2} \rb = 4 \pi \underset{\abs{r}}{\overset{\infty}{\int}} dr r^2  w_v \lb r \rb
\end{equation}

\todo{TODO}

\subsection{Alternative FMT functionals}
For the White Bear functional \cite{roth2002}, the bulk phase
properties are consistent with additive hard-sphere mixture
compressibillity of \citet{boublik1970} and
Mansoori-Carnahan-Starling-Leland (MCSL) \cite{mansoori1971}.

The BMCSL equation of state leads to a excess free energy density that
is slightly inconsistent, and a new generalization of the Carnahan-
Starling \cite{carnahan1969} equation of state to mixtures was derived,
the White Bear Mark II \cite{hansen-goos2006a}.

\subsection{The White Bear functional}

\begin{align}
  \Phi^\WB =& -n_0 \ln \lb 1 - n_3 \rb +
              \frac{n_1 n_2 - \vc{n}_1 \cdot \vc{n}_2}{1 - n_3}\nonumber \\
  &+ \lb n_2^3 - 3 n_2 \vc{n}_2 \cdot \vc{n}_2 \rb \frac{ n_3 + \lb 1 - n_3 \rb ^2 \ln \lb 1 - n_3 \rb}{36\pi n_3^2 \lb 1 - n_3 \rb^2}
\end{align}

\begin{align}
  \pd{\Phi^\WB}{n_0} &= -\ln \lb 1 - n_3 \rb \\
  \pd{\Phi^\WB}{n_1} &= \frac{n_2}{1 - n_3} \\
  \pd{\Phi^\WB}{n_2} &= \frac{n_1}{1 - n_3} + \lb n_2^2 - \vc{n}_2 \cdot \vc{n}_2 \rb \frac{ n_3 + \lb 1 - n_3 \rb ^2 \ln \lb 1 - n_3 \rb}{12\pi n_3^2 \lb 1 - n_3 \rb^2} \\
  \pd{\Phi^\WB}{n_3} &= \frac{n_0}{1 - n_3} +
  \frac{n_1 n_2 - \vc{n}_1 \cdot \vc{n}_2}{\lb 1 - n_3 \rb^2} \nonumber \\
  & + \lb n_2^3 - 3 n_2 \vc{n}_2 \cdot \vc{n}_2 \rb \biggl( \frac{n_3 \lb 5 - n_3 \rb - 2}{36\pi n_3^2 \lb 1 - n_3 \rb^3} - \frac{\ln \lb 1 - n_3 \rb}{18\pi n_3^3} \biggr) \\
  \pd{\Phi^\WB}{\vc{n}_1} &=  - \frac{\vc{n}_2}{1 - n_3} \\
  \pd{\Phi^\WB}{\vc{n}_2} &=  -\frac{\vc{n}_1}{1 - n_3} - n_2 \vc{n}_2 \frac{ n_3 + \lb 1 - n_3 \rb ^2 \ln \lb 1 - n_3 \rb}{6\pi n_3^2 \lb 1 - n_3 \rb^2}
\end{align}

\subsection{The White Bear Mark II functional}
\begin{align}
  \Phi^\WBII =& -n_0 \ln \lb 1 - n_3 \rb +
  \lb n_1 n_2 - \vc{n}_1 \cdot \vc{n}_2 \rb \frac{1 + \frac{1}{3} \phi_2\lb n_3 \rb}{1 - n_3} \nonumber \\
  &+ \lb n_2^3 - 3 n_2 \vc{n}_2 \cdot \vc{n}_2 \rb\frac{ 1 - \frac{1}{3} \phi_3\lb n_3 \rb }{24\pi n_3^2 \lb 1 - n_3 \rb^2}
\end{align}
with,
\begin{align}
  \phi_2\lb n_3 \rb =& \frac{1}{n_3}\lb 2 n_3 - n_3^2 + 2\lb 1-n_3 \rb\ln \lb 1 - n_3 \rb \rb \\
  \phi_3\lb n_3 \rb =& \frac{1}{n_3^2}\lb 2 n_3 - 3n_3^2 + 2n_3^3 + 2\lb 1-n_3 \rb^2\ln \lb 1 - n_3 \rb \rb
\end{align}

\begin{align}
  \od{\phi_2}{n_3} =&  - 1 - \frac{2}{n_3} - \frac{2\ln \lb 1 - n_3 \rb}{n_3^2} \\
  \od{\phi_3}{n_3} =& -\frac{4 (1 - n_3) \ln \lb 1 - n_3\rb}{n_3^3}  - \frac{4}{n_3^2} + \frac{2}{n_3} + 2
\end{align}


\begin{align}
  \pd{\Phi^\WBII}{n_0} &= -\ln \lb 1 - n_3 \rb \\
  \pd{\Phi^\WBII}{n_1} &= \frac{n_2 \lb 1 + \frac{1}{3} \phi_2 \rb}{1 - n_3} \\
  \pd{\Phi^\WBII}{n_2} &= \frac{n_1 \lb 1 + \frac{1}{3} \phi_2 \rb}{1 - n_3} + \frac{\lb n_2^2 - \vc{n}_2 \cdot \vc{n}_2 \rb \lb 1 - \frac{1}{3} \phi_3 \rb}{8\pi \lb 1 - n_3 \rb^2} \\
  \pd{\Phi^\WBII}{n_3} &= \frac{n_0}{1 - n_3} +
                         \lb n_1 n_2 - \vc{n}_1 \cdot \vc{n}_2 \rb \biggl( \frac{\frac{1}{3} \od{\phi_2}{n_3}}{1 - n_3}  + \frac{1 + \frac{1}{3} \phi_2}{\lb 1 - n_3 \rb^2} \biggr) \nonumber \\
                         &+\frac{\lb n_2^3 - 3 n_2 \vc{n}_2 \cdot \vc{n}_2\rb}{24\pi n_3^2 \lb 1 - n_3 \rb^2} \biggl( - \frac{1}{3} \od{\phi_3}{n_3}  + \biggl[\frac{ 1}{1 - n_3} - \frac{1}{n_3}\biggr] 2 \lb 1 - \frac{1}{3} \phi_3 \rb \biggr) \\
  \pd{\Phi^\WBII}{\vc{n}_1} &=  - \frac{\vc{n}_2 \lb 1 + \frac{1}{3} \phi_2 \rb}{1 - n_3} \\
  \pd{\Phi^\WBII}{\vc{n}_2} &=  - \frac{\vc{n}_1 \lb 1 + \frac{1}{3} \phi_2 \rb}{1 - n_3} - \frac{n_2 \vc{n}_2\lb 1 - \frac{1}{3} \phi_3 \rb}{4\pi n_3^2 \lb 1 - n_3 \rb^2}
\end{align}

\clearpage
\bibliographystyle{plainnat}
\bibliography{./DFT.bib,./HardSphere.bib}

\end{document}

%%% Local Variables: 
%%% mode: latex
%%% TeX-master: t
%%% ispell-local-dictionary: "british"
%%% End: 
