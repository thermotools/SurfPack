\documentclass[12pt, letterpaper]{article}
\pdfminorversion=4
\usepackage[utf8]{inputenc}
\usepackage{textcomp}
\usepackage{listings}
\usepackage{babel}
\usepackage[square,numbers]{natbib}
%\usepackage{amsmath}
\usepackage{amssymb,amsmath,amsthm,amsfonts}
\usepackage{pslatex}
%\usepackage{bm}% better \boldsymbol
\usepackage{array}% improves tabular environment.
\usepackage{dcolumn}% also improves tabular environment, with decimal centring.
%\usepackage{lastpage}
\usepackage{listings}
\usepackage{tikz}
\usetikzlibrary{decorations.pathmorphing}
%\usepackage{pgfplots}
%\usepackage{cleveref}
\usepackage{pgf,pgfarrows,pgfnodes}

\usepackage[final]{pdfpages}
\usepackage{hyperref}
\usepackage{doi}

%\usepackage[altbullet,expert]{lucidabr}
%\usepackage[charter]{mathdesign}
\usepackage{cleveref}
\usepackage{ifthen}
%\usepackage{nomencl}
\usepackage{parskip}
\usepackage{psfrag}
\usepackage{booktabs}
\usepackage{pifont}
%\usepackage{subfig,caption}
\usepackage{microtype}% for pdflatex; medisin mot overfull \vbox
\usepackage{xspace}% for pdflatex; medisin mot overfull \vbox
\usepackage[font=footnotesize]{subcaption}
\usepackage{siunitx}
\usepackage{todonotes}
\presetkeys%
    {todonotes}%
    {inline,backgroundcolor=orange}{}

%% Formattering av figur- og tabelltekster
%
%
% Egendefinerte
%
\newcommand*{\unit}[1]{\ensuremath{\,\mathrm{#1}}}
\newcommand*{\uunit}[1]{\ensuremath{\mathrm{#1}}}
%\newcommand*{\od}[3][]{\frac{\mathrm{d}^{#1}#2}{\mathrm{d}{#3}^{#1}}}% ordinary derivative
\newcommand*{\od}[3][]{\frac{\dif^{#1}#2}{\dif{#3}^{#1}}}% ordinary derivative
\newcommand*{\pd}[3][]{\frac{\partial^{#1}#2}{\partial{#3}^{#1}}}% partial derivative
\newcommand*{\pdt}[3][]{{\partial^{#1}#2}/{\partial{#3}^{#1}}}% partial
                                % derivative for inline use.
\newcommand{\pone}[3]{\frac{\partial #1}{\partial #2}_{#3}}% partial
                                % derivative with information of
                                % constant variables
\newcommand{\ponel}[3]{\frac{\partial #1}{\partial #2}\bigg|_{#3}} % partial derivative with informatio of constant variable. A line is added.
\newcommand{\ptwo}[3]{\frac{\partial^{2} #1}{\partial #2 \partial
    #3}} % partial differential in two different variables
\newcommand{\pdn}[3]{\frac{\partial^{#1}#2}{\partial{#3}^{#1}}}% partial derivative

% Total derivative:
\newcommand*{\ttd}[2]{\frac{\mathrm{D} #1}{\mathrm{D} #2}}
\newcommand*{\td}[2]{\frac{\mathrm{d} #1}{\mathrm{d} #2}}
\newcommand*{\ddt}{\frac{\partial}{\partial t}}
\newcommand*{\ddx}{\frac{\partial}{\partial x}}
% Vectors etc:
% For Computer Modern:

\DeclareMathAlphabet{\mathsfsl}{OT1}{cmss}{m}{sl}
%\renewcommand*{\vec}[1]{\boldsymbol{#1}}%
\newcommand*{\vc}[1]{\vec{\mathbf{#1}}}%
\newcommand*{\tensor}[1]{\mathsfsl{#1}}% 2. order tensor
\newcommand*{\matr}[1]{\tensor{#1}}% matrix
\renewcommand*{\div}{\boldsymbol{\nabla\cdot}}% divergence
\newcommand*{\grad}{\boldsymbol{\nabla}}% gradient
% fancy differential from Claudio Beccari, TUGboat:
% adjusts spacing automatically
\makeatletter
\newcommand*{\dif}{\@ifnextchar^{\DIfF}{\DIfF^{}}}
\def\DIfF^#1{\mathop{\mathrm{\mathstrut d}}\nolimits^{#1}\gobblesp@ce}
\def\gobblesp@ce{\futurelet\diffarg\opsp@ce}
\def\opsp@ce{%
  \let\DiffSpace\!%
  \ifx\diffarg(%
    \let\DiffSpace\relax
  \else
    \ifx\diffarg[%
      \let\DiffSpace\relax
    \else
      \ifx\diffarg\{%
        \let\DiffSpace\relax
      \fi\fi\fi\DiffSpace}
\makeatother
%
\newcommand*{\me}{\mathrm{e}}% e is not a variable (2.718281828...)
%\newcommand*{\mi}{\mathrm{i}}%  nor i (\sqrt{-1})
\newcommand*{\mpi}{\uppi}% nor pi (3.141592...) (works for for Lucida)
%
% lav tekst-indeks/subscript/pedex
\newcommand*{\ped}[1]{\ensuremath{_{\text{#1}}}}
% hy tekst-indeks/superscript/apex
\newcommand*{\ap}[1]{\ensuremath{^{\text{#1}}}}
\newcommand*{\apr}[1]{\ensuremath{^{\mathrm{#1}}}}
\newcommand*{\pedr}[1]{\ensuremath{_{\mathrm{#1}}}}
%
\newcommand*{\volfrac}{\alpha}% volume fraction
\newcommand*{\surften}{\sigma}% coeff. of surface tension
\newcommand*{\curv}{\kappa}% curvature
\newcommand*{\ls}{\phi}% level-set function
\newcommand*{\ep}{\Phi}% electric potential
\newcommand*{\perm}{\varepsilon}% electric permittivity
\newcommand*{\visc}{\mu}% molecular (dymamic) viscosity
\newcommand*{\kvisc}{\nu}% kinematic viscosity
\newcommand*{\cfl}{C}% CFL number

% Grid
\newcommand{\jj}{j}
\newcommand{\jph}{{j+1/2}}
\newcommand{\jmh}{{j-1/2}}
\newcommand{\jp}{{j+1}}
\newcommand{\jm}{{j-1}}
\newcommand{\nn}{n}
\newcommand{\nph}{{n+1/2}}
\newcommand{\nmh}{{n-1/2}}
\newcommand{\np}{{n+1}}
%
\newcommand{\lf}{\text{LF}}
\newcommand{\lw}{\text{Ri}}
%
\newcommand*{\cons}{\vec U}
\newcommand*{\flux}{\vec F}
\newcommand*{\dens}{\rho}
\newcommand*{\svol}{\ensuremath v}
\newcommand*{\temp}{\ensuremath T}
\newcommand*{\vel}{\ensuremath u}
\newcommand*{\mom}{\dens\vel}
\newcommand*{\toten}{\ensuremath E}
\newcommand*{\inten}{\ensuremath e}
\newcommand*{\press}{\ensuremath p}
\renewcommand*{\ss}{\ensuremath a}
\newcommand*{\jac}{\matr A}
%
\newcommand*{\abs}[1]{\lvert#1\rvert}
\newcommand*{\bigabs}[1]{\bigl\lvert#1\bigr\rvert}
\newcommand*{\biggabs}[1]{\biggl\lvert#1\biggr\rvert}
\newcommand*{\norm}[1]{\lVert#1\rVert}
%
\newcommand*{\e}[1]{\times 10^{#1}}
\newcommand*{\ex}[1]{\times 10^{#1}}%shorthand -- for use e.g. in tables
\newcommand*{\exi}[1]{10^{#1}}%shorthand -- for use e.g. in tables
\newcommand*{\nondim}[1]{\ensuremath{\mathit{#1}}}% italic iflg. ISO. (???)
\newcommand*{\rey}{\nondim{Re}}
\newcommand*{\acro}[1]{\textsc{\MakeLowercase{#1}}}%acronyms etc.

\newcommand{\nto}{\ensuremath{\mbox{N}_{\mbox{\scriptsize 2}}}}
\newcommand{\chfire}{\ensuremath{\mbox{CH}_{\mbox{\scriptsize 4}}}}
%\newcommand*{\checked}{\ding{51}}
\newcommand{\coto}{\ensuremath{\text{CO}_{\text{\scriptsize 2}}}}
\newcommand{\celsius}{\ensuremath{^\circ\text{C}}}
%\newcommand{\clap}{Clapeyron~}
\newcommand{\subl}{\ensuremath{\text{sub}}}
\newcommand{\spec}{\text{spec}}
\newcommand{\sat}{\text{sat}}
\newcommand{\sol}{\text{sol}}
\newcommand{\liq}{\text{liq}}
\newcommand{\vap}{\text{vap}}
\newcommand{\amb}{\text{amb}}
\newcommand{\tr}{\text{tr}}
\newcommand{\crit}{\text{crit}}
\newcommand{\entr}{\ensuremath{\text{s}}}
\newcommand{\fus}{\text{fus}}
\newcommand{\flash}[1]{\ensuremath{#1\text{-flash}}}
\newcommand{\spce}[2]{\ensuremath{#1\, #2\text{ space}}}
\newcommand{\spanwagner}{\text{Span--Wagner}}
\newcommand{\triplepoint}{\text{TP triple point}}
\newcommand{\wrpt}{\text{with respec to~}}
%\sisetup{input-symbols = {( )}}
% \newcommand*{\red}{\ensuremath{\text{R}}\xspace}
% \newcommand*{\crit}{\ensuremath{\text{c}}\xspace}
% \newcommand*{\mix}{\ensuremath{\text{m}}\xspace}
\newcommand*{\lb}{\ensuremath{\left(}}
\newcommand*{\rb}{\ensuremath{\right)}}
\newcommand*{\lbf}{\ensuremath{\left[}}
\newcommand*{\rbf}{\ensuremath{\right]}}
\newcommand{\LEFT}{\ensuremath{\text{L}}\xspace}
\newcommand{\RIGTH}{\ensuremath{\text{R}}\xspace}
\newcommand{\cdft}{\ensuremath{\text{classical DFT}}\xspace}
\newcommand{\RF}{\ensuremath{\text{RF}}\xspace}
\newcommand{\WB}{\ensuremath{\text{WB}}\xspace}
\newcommand{\WBII}{\ensuremath{\text{WBII}}\xspace}
\newcommand{\excess}{\ensuremath{\text{ex}}\xspace}
\newcommand{\ideal}{\ensuremath{\text{id}}\xspace}
\newcommand{\rvec}{\ensuremath{\mathbf{r}}\xspace}
\newcommand{\pvec}{\ensuremath{\mathbf{p}}\xspace}
\newcommand{\attractive}{\ensuremath{\text{att}}\xspace}
\newcommand{\kB}{\ensuremath{\text{k}_{\text{B}}}\xspace}
\newcommand{\external}{\ensuremath{\text{ext}}\xspace}
\newcommand{\bulk}{\ensuremath{\text{b}}\xspace}
\newcommand{\pure}{\ensuremath{\text{p}}\xspace}
\newcommand{\NC}{\ensuremath{\text{NC}}\xspace}

\graphicspath{{gfx/}}

\title{Classical Density Functional Theory (cDFT) for Thermopack}
\author{Morten Hammer}
\date{\today}

\begin{document}
%\tableofcontents
%\printnomenclature


\begin{titlepage}
\maketitle
\end{titlepage}

\section{Introduction}

The Jupiter notebooks of Mary K. Coe
\href{https://github.com/marykcoe/cDFT_Package}{cDFT} is a great recourse for
understanding \cdft. Her PhD thesis also contsin's a lot of information
\cite{coe2021}.

\section{Fundamental Measure Theory}
Fundamental measure theory for hard sphere mixtures was developed by
\citet{rosenfeld1989}. The name "measure" relates to the fundamental
geometrical measures (volume, surface area, mean radius of curvature
and the Euler characteristic) of a sphere particle. The fundamental
geometrical measures are recovered when integrating the weight
functions defined in Section~\ref{sec:weight}.

For bulk phases this functional reduces to the Pick's-Yevick (PY)
compressibility equation \cite{percus1958}, equivalent to scaled
particle theory.

\subsection{The Rosenfeld functional}

The functional depends on the weighted densities,
\begin{equation}
  n_\alpha = \int d \mathbf{r}^\prime \rho \lb \mathbf{r}^\prime \rb w_\alpha \lb \mathbf{r} - \mathbf{r}^\prime \rb.
\end{equation}

\begin{equation}
  \Phi^\RF = -n_0 \ln \lb 1 - n_3 \rb +
  \frac{n_1 n_2 - \vc{n}_1 \cdot \vc{n}_2}{1 - n_3} +
  \frac{n_2^3 - 3 n_2 \vc{n}_2 \cdot \vc{n}_2}{24\pi \lb 1 - n_3 \rb^2}
\end{equation}

The differentials needed when searching for the Grand potential and the equilibrium density profile:
\begin{align}
  \pd{\Phi^\RF}{n_0} &= -\ln \lb 1 - n_3 \rb \\
  \pd{\Phi^\RF}{n_1} &= \frac{n_2}{1 - n_3} \\
  \pd{\Phi^\RF}{n_2} &= \frac{n_1}{1 - n_3} + \frac{n_2^2 - \vc{n}_2 \cdot \vc{n}_2}{8\pi \lb 1 - n_3 \rb^2} \\
  \pd{\Phi^\RF}{n_3} &= \frac{n_0}{1 - n_3} +
  \frac{n_1 n_2 - \vc{n}_1 \cdot \vc{n}_2}{\lb 1 - n_3 \rb^2} +
  \frac{n_2^3 - 3 n_2 \vc{n}_2 \cdot \vc{n}_2}{12\pi \lb 1 - n_3 \rb^3} \\
  \pd{\Phi^\RF}{\vc{n}_1} &=  - \frac{\vc{n}_2}{1 - n_3} \\
  \pd{\Phi^\RF}{\vc{n}_2} &=  -\frac{\vc{n}_1}{1 - n_3} - \frac{n_2 \vc{n}_2}{4\pi \lb 1 - n_3 \rb^2}
\end{align}

\subsection{Weight functions}
\label{sec:weight}
Weight functions given by

\begin{align}
  w_3^i \lb \mathbf{r} \rb &=  \Theta \lb R_i - \abs{\mathbf{r}} \rb \\
  w_2^i \lb \mathbf{r} \rb &=  \delta \lb R_i - \abs{\mathbf{r}} \rb \\
  w_1^i \lb \mathbf{r} \rb &= \frac{1}{4 \pi R_i} w_2^i \lb \mathbf{r} \rb \\
  w_0^i \lb \mathbf{r} \rb &= \frac{1}{4 \pi R_i^2} w_2^i \lb \mathbf{r} \rb \\
  \mathbf{w}_2^i \lb \mathbf{r} \rb &=  \frac{\mathbf{r}}{\abs{\mathbf{r}}}\delta \lb R_i - \abs{\mathbf{r}} \rb \\
  \mathbf{w}_1^i \lb \mathbf{r} \rb &= \frac{1}{4 \pi R_i} \mathbf{w}_2^i.
\end{align}
Where $\Theta$ is the Heaviside function, and $\delta$ are the Dirac delta function.

\subsubsection{Weight functions for planar geometry}
\label{sec:planar_weights}
For the planar geometry $\rho \lb \mathbf{r} \rb = \rho \lb z \rb $,
and the weigth functions can be integrated for the $x,y$ dimensions.

\begin{equation}
  W_v \lb z \rb = \underset{-\infty}{\overset{\infty}{\int}} \underset{-\infty}{\overset{\infty}{\int}} dx dy w_v \lb \sqrt{x^2 + y^2 + z^2} \rb = 2 \pi \underset{\abs{z}}{\overset{\infty}{\int}} dr r  w_v \lb r \rb
\end{equation}

This can be integrated analytically to
\begin{align}
  w_3^i \lb z \rb &=  \pi \lb R_i^2 - z^2 \rb \Theta \lb R_i - \abs{z} \rb \\
  w_2^i \lb z \rb &=  2 \pi R_i \Theta \lb R_i - \abs{z} \rb \\
  \mathbf{w}_2^i \lb z \rb &= 2 \pi z \mathbf{e}_z  \Theta \lb R_i - \abs{z} \rb
\end{align}

The planar weight functions are visualized in Figure
\ref{fig:planar_weights}.
\begin{figure}[tbp]
  \centering
  \includegraphics[width=0.49\textwidth]{gfx/planar_weights}
  \caption{Planar weight functions.}
  \label{fig:planar_weights}
\end{figure}

\subsubsection{Weight functions for spherical geometry}

For the shperical geometry $\rho \lb \mathbf{r} \rb = \rho \lb r \rb $,
and the weigth functions can be integrated for the angle dimensions.

\begin{equation}
  W_v \lb r \rb = \underset{-\infty}{\overset{\infty}{\int}} \underset{-\infty}{\overset{\infty}{\int}} dx dy w_v \lb \sqrt{x^2 + y^2 + z^2} \rb = 4 \pi \underset{\abs{r}}{\overset{\infty}{\int}} dr r^2  w_v \lb r \rb
\end{equation}

\todo{TODO}

\subsection{The one body correlation function}
The one body correlation functions is given from the Helmholtz free energy functional as,

\begin{equation}
  c^{(1)}\lb \mathbf{r} \rb = \beta \pd{\mathcal{F}_{\excess} \lbf \rho \rbf}{\rho \lb \mathbf{r} \rb} =  -\underset{\alpha}{\sum} \int d \mathbf{r}^\prime \pd{\Phi_\alpha}{n_\alpha} \pd{n_\alpha}{\rho}.
\end{equation}

In a planar geometry, the one body correlation function simply becomes,
\begin{equation}
  \label{eq:dndrho}
  \pd{n_\alpha \lb z^\prime \rb}{\rho \lb z \rb} = \frac{\partial}{\partial \rho \lb z \rb} \int dz^{\prime\prime} \rho \lb z^{\prime\prime} \rb w_\alpha \lb z^\prime - z^{\prime\prime} \rb = w_\alpha \lb z^\prime - z\rb,
\end{equation}
\begin{equation}
  \label{eq:c1_1d}
  c^{(1)}\lb z \rb = -\underset{\alpha}{\sum} \int d z^\prime \pd{\Phi_\alpha}{n_\alpha} w_\alpha \lb z^\prime - z\rb.
\end{equation}

\subsection{Alternative FMT functionals}
For the White Bear functional \cite{roth2002}, the bulk phase
properties are consistent with additive hard-sphere mixture
compressibillity of \citet{boublik1970} and
Mansoori-Carnahan-Starling-Leland (MCSL) \cite{mansoori1971}.

The BMCSL equation of state leads to a excess free energy density that
is slightly inconsistent, and a new generalization of the Carnahan-
Starling \cite{carnahan1969} equation of state to mixtures was derived,
the White Bear Mark II \cite{hansen-goos2006a}.

\subsection{The White Bear functional}

\begin{align}
  \Phi^\WB =& -n_0 \ln \lb 1 - n_3 \rb +
              \frac{n_1 n_2 - \vc{n}_1 \cdot \vc{n}_2}{1 - n_3}\nonumber \\
  &+ \lb n_2^3 - 3 n_2 \vc{n}_2 \cdot \vc{n}_2 \rb \frac{ n_3 + \lb 1 - n_3 \rb ^2 \ln \lb 1 - n_3 \rb}{36\pi n_3^2 \lb 1 - n_3 \rb^2}
\end{align}

\begin{align}
  \pd{\Phi^\WB}{n_0} &= -\ln \lb 1 - n_3 \rb \\
  \pd{\Phi^\WB}{n_1} &= \frac{n_2}{1 - n_3} \\
  \pd{\Phi^\WB}{n_2} &= \frac{n_1}{1 - n_3} + \lb n_2^2 - \vc{n}_2 \cdot \vc{n}_2 \rb \frac{ n_3 + \lb 1 - n_3 \rb ^2 \ln \lb 1 - n_3 \rb}{12\pi n_3^2 \lb 1 - n_3 \rb^2} \\
  \pd{\Phi^\WB}{n_3} &= \frac{n_0}{1 - n_3} +
  \frac{n_1 n_2 - \vc{n}_1 \cdot \vc{n}_2}{\lb 1 - n_3 \rb^2} \nonumber \\
  & + \lb n_2^3 - 3 n_2 \vc{n}_2 \cdot \vc{n}_2 \rb \biggl( \frac{n_3 \lb 5 - n_3 \rb - 2}{36\pi n_3^2 \lb 1 - n_3 \rb^3} - \frac{\ln \lb 1 - n_3 \rb}{18\pi n_3^3} \biggr) \\
  \pd{\Phi^\WB}{\vc{n}_1} &=  - \frac{\vc{n}_2}{1 - n_3} \\
  \pd{\Phi^\WB}{\vc{n}_2} &=  -\frac{\vc{n}_1}{1 - n_3} - n_2 \vc{n}_2 \frac{ n_3 + \lb 1 - n_3 \rb ^2 \ln \lb 1 - n_3 \rb}{6\pi n_3^2 \lb 1 - n_3 \rb^2}
\end{align}

\subsection{The White Bear Mark II functional}
\begin{align}
  \Phi^\WBII =& -n_0 \ln \lb 1 - n_3 \rb +
  \lb n_1 n_2 - \vc{n}_1 \cdot \vc{n}_2 \rb \frac{1 + \frac{1}{3} \phi_2\lb n_3 \rb}{1 - n_3} \nonumber \\
  &+ \lb n_2^3 - 3 n_2 \vc{n}_2 \cdot \vc{n}_2 \rb\frac{ 1 - \frac{1}{3} \phi_3\lb n_3 \rb }{24\pi \lb 1 - n_3 \rb^2}
\end{align}
with,
\begin{align}
  \phi_2\lb n_3 \rb =& \frac{1}{n_3}\lb 2 n_3 - n_3^2 + 2\lb 1-n_3 \rb\ln \lb 1 - n_3 \rb \rb \\
  \phi_3\lb n_3 \rb =& \frac{1}{n_3^2}\lb 2 n_3 - 3n_3^2 + 2n_3^3 + 2\lb 1-n_3 \rb^2\ln \lb 1 - n_3 \rb \rb
\end{align}

\begin{align}
  \od{\phi_2}{n_3} =&  - 1 - \frac{2}{n_3} - \frac{2\ln \lb 1 - n_3 \rb}{n_3^2} \\
  \od{\phi_3}{n_3} =& -\frac{4 (1 - n_3) \ln \lb 1 - n_3\rb}{n_3^3}  - \frac{4}{n_3^2} + \frac{2}{n_3} + 2
\end{align}


\begin{align}
  \pd{\Phi^\WBII}{n_0} &= -\ln \lb 1 - n_3 \rb \\
  \pd{\Phi^\WBII}{n_1} &= \frac{n_2 \lb 1 + \frac{1}{3} \phi_2 \rb}{1 - n_3} \\
  \pd{\Phi^\WBII}{n_2} &= \frac{n_1 \lb 1 + \frac{1}{3} \phi_2 \rb}{1 - n_3} + \frac{\lb n_2^2 - \vc{n}_2 \cdot \vc{n}_2 \rb \lb 1 - \frac{1}{3} \phi_3 \rb}{8\pi \lb 1 - n_3 \rb^2} \\
  \pd{\Phi^\WBII}{n_3} &= \frac{n_0}{1 - n_3} +
                         \lb n_1 n_2 - \vc{n}_1 \cdot \vc{n}_2 \rb \biggl( \frac{\frac{1}{3} \od{\phi_2}{n_3}}{1 - n_3}  + \frac{1 + \frac{1}{3} \phi_2}{\lb 1 - n_3 \rb^2} \biggr) \nonumber \\
                         &+\frac{\lb n_2^3 - 3 n_2 \vc{n}_2 \cdot \vc{n}_2\rb}{24\pi \lb 1 - n_3 \rb^2} \biggl( - \frac{1}{3} \od{\phi_3}{n_3}  + \frac{ 2  \lb 1 - \frac{1}{3} \phi_3 \rb }{1 - n_3}\biggr) \\
  \pd{\Phi^\WBII}{\vc{n}_1} &=  - \frac{\vc{n}_2 \lb 1 + \frac{1}{3} \phi_2 \rb}{1 - n_3} \\
  \pd{\Phi^\WBII}{\vc{n}_2} &=  - \frac{\vc{n}_1 \lb 1 + \frac{1}{3} \phi_2 \rb}{1 - n_3} - \frac{n_2 \vc{n}_2\lb 1 - \frac{1}{3} \phi_3 \rb}{4\pi \lb 1 - n_3 \rb^2}
\end{align}

\section{Numerics}

Solving of the convolution integrals in the FMT and cDFT in real space
uses $\mathcal{O}\lb N^2 \rb$ operations, however according to the
convolution theorem the integrals can be done by Fourier
transformations, leading to only $\mathcal{O}\lb N \ln N \rb$
operations \cite{roth2010, knepley2010}. Different options for solving
the discrete fast Fourier transform (FFT) is available, FFTW (GNU
General Public License), FFTPACK (MIT) and Python FFT.

The common approach used when solving \cdft problems is Picard
iterations. Instead of using a successive substituting iteration,
$\tilde{\rho}^{(i)} \rightarrow \rho^{(i)}$, a mixing of the new
density with the original density is used to dampen the effect of the
new value, accordign to,
\begin{equation}
  \tilde{\rho}^{(i+1)}\lb z \rb = \alpha \rho^{(i)}\lb z \rb + \lb 1 - \alpha \rb \tilde{\rho}^{(i)}\lb z \rb.
\end{equation}
The main reason is to avoid $n_3$ values exceeding unity.

Often the Picard parameter is set to a fixed low value, typically
$\alpha=0.1$, resulting in slow convergence. However using a line
search requiring some decay in error is probably the best way to
implement the Picard iterations. \citet{roth2010} suggest using a
simple quadratic line search. \citet{roth2010} used the Grand
Potential $\Omega$, when evaluating the line
search. \cite{knepley2010} evaluated the
$\| \tilde{\rho}^{(i)} - \rho^{(i)} \|$ as a function of $\alpha$ and
found the minimum of a quadratic polynomial.

One simple way of accelerate the solution of the equilibrium density
profile is by extrapolation as used by \citet{ng1974}.

\cite{knepley2010} tested a Newton solver (using numerical
approximations for the differentials), but they report linear
convergence through most of the iteration steps. The use of
inefficient generation of differentials was also reported as an issue.

Looking at Equation \eqref{eq:dndrho} and \eqref{eq:c1_1d} we see that
differentiating Equation \eqref{eq:c1_1d} will require convolution of
the $\Phi_{\alpha\gamma}$ with
$w_\alpha \lb z^\prime - z_1\rb w_\gamma \lb z^\prime - z_2\rb$. The
latter will become a matrix constant matrix requiering a convolution
integral per element in the banded Jacobian. The matrix is constant
and only the inverse Fourier transform will require computuational
effort. Each of these elements will require CPU time similar to one
half Picard iteration. For example if there are 1000 grid cells over
the diameter of a particle, the generation of one Jacobian instance
will be similar to 500 Picard iterations.

Parallel solution for the Fourier transforms are simple using the FFTW library....

\subsubsection{Quadratures for the weight functions}
Integrating on a regular grid the integral can be made more accurate using a qudrature formula \todo{Cite},
\begin{align}
  \underset{z_N}{\overset{z_1}{\int}} dz^\prime f \lb z^\prime \rb g \lb z^\prime - z \rb =& \Delta z \biggl( \frac{3}{8} f_1 g_{i-1} + \frac{7}{6} f_2 g_{i-2} + \frac{23}{24} f_3 g_{i-3} + f_4 g_{i-4} \nonumber \\&+ \dots  + f_{N-3} g_{i-N+3} + \frac{23}{24} f_{N-2} g_{i-N+2} \nonumber \\&+ \frac{7}{6} f_{N-1} g_{i-N+1} + \frac{3}{8} f_{N-2} g_{i-N+2}.
\end{align}
The qudarature is implemeted by multipying the end weights with by the
quadrature weights.

The actual planar weight functions are visualized in Figure
\ref{fig:actual_planar_weights}.
\begin{figure}[tbp]
  \centering
  \includegraphics[width=0.49\textwidth]{gfx/actual_planar_weights}
  \caption{Actual planar weight functions.}
  \label{fig:actual_planar_weights}
\end{figure}

\section{Perturbation theory}
The canonical partition function,
\begin{equation}
  Q_N = \frac{1}{h^{3N}N!}\int d \mathbf{p}^N \int d \rvec^N e^{-\beta \mathcal{H}}
\end{equation}
Relation between Helmholtz energy and partition function,
\begin{equation}
  \label{eq:helm_statmec}
  F = -\kB T \ln Q_N
\end{equation}
Using a Hamiltonian,
\begin{equation}
  \mathcal{H} = \Phi \lb \rvec^N \rb + K\lb \pvec^N \rb + V_\external \lb \rvec^N \rb,
\end{equation}
where the kinetic energies is given from the moments,
\begin{equation}
  K\lb \pvec^N \rb = \overset{N}{\underset{i=1}{\sum}} \frac{\abs{\pvec_i}^2}{2 m}
\end{equation}
the partition function can be integrated with respect to the moments,
\begin{align}
  Q_N &= \frac{1}{h^{3N}N!}\int d \mathbf{p}^N e^{-beta + K\lb \pvec^N \rb } \int d \rvec^N e^{-\beta \lb \Phi \lb \rvec^N \rb + V_\external \lb \rvec^N \rb \rb} \nonumber \\
      &= \frac{1}{\Lambda^{3N}N!} \int d \rvec^N e^{-\beta \lb \Phi \lb \rvec^N \rb + V_\external \lb \rvec^N \rb \rb} \nonumber\\
  &= \frac{Z_N}{\Lambda^{3N}N!}
\end{align}
where $Z_N$ is the configurational integral, and $\Lambda$ is the
thermal de Broglie wavelength.

Having the perturbation potential
\begin{equation}
  \phi_\lambda \lb \rvec, \rvec^\prime \rb = \phi_0 \lb \rvec, \rvec^\prime \rb + \lambda \phi_\attractive \lb \rvec, \rvec^\prime \rb \quad 0 \le \lambda \le 1,
\end{equation}
where $\lambda$ is the perturbation strength, the potential energy felt between all particles is given by
\begin{equation}
\Phi \lb \rvec^N \rb = \overset{N}{\underset{j=1}{\sum}} \overset{N}{\underset{k>j}{\sum}} \phi_\lambda \lb \rvec, \rvec^\prime \rb
\end{equation}
The excess Helmholtz energy can be differentiated with respect to $\lambda$ using Equation \eqref{eq:helm_statmec},
\begin{equation}
  \label{eq:helm_expancion}
\beta \pd{F_\excess}{\lambda} = -\frac{1}{Z_N} \pd{Z_N}{\lambda} = \frac{\beta}{2} \int d \rvec \int d \rvec^\prime \rho_\lambda^{(2)}\lb \rvec, \rvec^\prime \rb \phi_\attractive \lb \rvec, \rvec^\prime \rb
\end{equation}
we can also describe the Helmholtz energy using ensemble average, $\langle \dots \rangle_\lambda$, for a system described by $\phi_\lambda$,
\begin{equation}
\beta \pd{F_\excess}{\lambda} = \langle \Phi^\prime \rangle_\lambda
\end{equation}
where $\pd{\Phi_\lambda}{\lambda} = \Phi_\lambda^\prime$. Integration yields,
\begin{equation}
\beta F_\excess = \beta F_0 + \overset{\lambda=1}{\underset{\lambda=0}{\int}} d \lambda \langle \Phi^\prime \rangle_\lambda
\end{equation}

In order to get ensemble averages over the reference system
$\lambda = 0$, the average can be expanded in $\lambda$ around
$\lambda = 0$.

Leading to
\begin{equation}
\beta F_\excess = \beta F_0 + \beta F_1 + \beta F_2 + \beta F_3 + \mathcal{O}\lb \beta^4 \rb
\end{equation}
where
\begin{align}
  \beta F_1 &= \beta \langle \Phi_\attractive  \rangle_0\\
  \beta F_2 &= -\frac{\beta^2}{2} \biggl[ \langle \Phi_\attractive^2  \rangle_0 - \langle \Phi_\attractive  \rangle_0^2 \biggr] \\
  \beta F_3 &= \frac{\beta^3}{3!} \bigg\langle \Phi_\attractive - \langle \Phi_\attractive  \rangle_0 \bigg\rangle^3
\end{align}

For pair-wise additive potentials we have,
\begin{equation}
  \label{eq:first_order_pert}
\frac{\beta F_1}{N} = \frac{\beta \rho}{2} \int  g_\lambda \lb r \rb \phi_\attractive \lb r \rb d r
\end{equation}
and to first order $g_\lambda = g_0$.

\section{Approaches used when extending \cdft to attractive fluids}

\subsection{Mean Field Theory (MFT)}
Under the MFT approximation, $g_ \approx 1$, and Equation
\eqref{eq:first_order_pert} simply becomes
\begin{equation}
  \label{eq:mft}
  \frac{\beta F_1}{N} = \frac{\beta \rho}{2} \phi_\attractive \lb r \rb d r
\end{equation}

For some reason it is common to use the WCA perturbation potential,
however the hard-sphere diameter seem to be independent of density.
\todo{Check if cDFT\_Package uses sigma=1 with WCA simulation....}

\subsection{Local density approximation (LDA)}
Under the LDA assumption the Helmholtz energy density of an
inhomogeneous system with density profile $\rho \lb r\rb$ is
calculated using the bulk phase Helmholtz energy density evaluated at
the value of the local density. This often work for surface tension
calculations, however adjacent to walls where the density oscillate
strongly and the local density can exceed the maximum packing fractions
this will be a problem.

\subsection{Weighted density approximation (WDA)}
The WDA uses locally weighted densities and evaluates the Helmholtz
energy functional with these densities. This methodology have proven
successful even for fluid to wall interacting systems.

\citet{sauer2017}
\citet{tarazona1984, tarazona1984a}

\subsection{Nonlocal perturbation theory (NLP)}
\citet{gloor2004}
\citet{gross2009}

\section{The PCP-SAFT \cdft}

\citet{sauer2017}

PC-SAFT \citet{gross2001}
Polar extensions
Quadruplole-Quadruplole:\citet{gross2005}
Dipole-dipole:\citet{gross2006}
Dipole-Quadruplole: \citet{vrabec2008}


\section{Analytial Fourier transforms of the weight functions}
\citet[Appendix B]{knepley2010} derives the analytical Fourier
transform for the weight functions.

\subsection{Planar geometry}
The weight functions in a planar geometry is derived in section
\ref{sec:planar_weights}. The weight functions can be transformed to
Fourier space according to the definition,
\begin{align}
  \label{eq:fourier}
  \hat{w}_\alpha^i \lb k \rb = \mathcal{F}\lb w_\alpha^i \lb z \rb \rb =& \overset{\infty}{\underset{-\infty}{\int}} dz w_\alpha^i \lb z \rb e^{-i k z} \nonumber \\
  =& \overset{\infty}{\underset{-\infty}{\int}} dz w_\alpha^i \lb z \rb \cos \lb k z \rb + i \overset{\infty}{\underset{-\infty}{\int}} dz w_\alpha^i \lb z \rb \sin \lb k z \rb
\end{align}

Since $w_3^i$ and $w_2^i$ are even functions, the Fourier transform
will be purely real valued, while $\mathbf{w}_2^i$ is odd and
therefore purly imaginary,
\begin{align}
  \hat{w}_3^i &=  \overset{\infty}{\underset{-\infty}{\int}} dz  \pi \lb R_i^2 - z^2 \rb \Theta \lb R_i - \abs{z} \rb \cos \lb k z \rb \nonumber \\
              &=  \pi \overset{R_i}{\underset{-R_i}{\int}} dz  \lb R_i^2 - z^2 \rb \cos \lb k z \rb  \nonumber \\
              &=  \frac{4 \pi}{k^3} \biggl( \sin \lb k R_i \rb - k R\cos \lb k R_i \rb \biggr) \label{eq:w_hat_3}
\end{align}

\begin{align}
  \hat{w}_2^i &=  \overset{\infty}{\underset{-\infty}{\int}} dz 2 \pi R_i \Theta \lb R_i - \abs{z} \rb \cos \lb k z \rb \nonumber \\
              &=  2 \pi R_i \overset{R_i}{\underset{-R_i}{\int}} dz  \cos \lb k z \rb  \nonumber \\
              &=  \frac{4 \pi R_i}{k} \sin \lb k R_i \rb \label{eq:w_hat_2}
\end{align}

\begin{align}
  \hat{\mathbf{w}}_2^i &=  i \overset{\infty}{\underset{-\infty}{\int}} dz 2 \pi \mathbf{z}  \Theta \lb R_i - \abs{z} \rb \cos \lb \mathbf{k} \cdot \mathbf{z} \rb \nonumber \\
                       &=  2 \pi i  \overset{R_i}{\underset{-R_i}{\int}} dz \mathbf{z}  \cos \lb \mathbf{k} \cdot \mathbf{z} \rb =  - 2 \pi i \mathbf{e}_k  \overset{R_i}{\underset{-R_i}{\int}} dz z  \cos \lb k z \rb \nonumber \\
                       &=  -\frac{4 \pi i}{k^2} \biggl( \sin \lb k R_i \rb - k R_i \cos \lb k R_i \rb \biggr) \mathbf{e}_k \label{eq:w_hat_2v}
\end{align}
Comparing equations \eqref{eq:w_hat_3}, \eqref{eq:w_hat_2} and \eqref{eq:w_hat_2v}, we see that the equation s


\section{Bulk properties for hard spheres}

The excess pressure of the system is described as
\begin{equation}
  \label{eq:pressure}
  \beta p_{\excess} = -\pd{\beta \mathcal{F_{\excess}}}{V} = -\pd{\lb V \Phi \rb}{V} = -\Phi - V\underset{i=1}{\sum} \pd{\Phi}{n_{\alpha}}\pd{n_{\alpha}}{V} = -\Phi + \underset{i=1}{\sum} \pd{\Phi}{n_{\alpha}}n_{\alpha}
\end{equation}

The ideal pressure of the system is simply
\begin{equation}
  \beta p_{\ideal} = n_{0}.
\end{equation}

The excess chemical potential of the system is described as
\begin{equation}
  \label{eq:chem_pot}
  \hat{\mu}^i_{\excess} = \beta \mu^i_{\excess} = \pd{\beta \mathcal{F_{\excess}}}{N_i} = \pd{\lb V \Phi \rb}{N_i} = \pd{\Phi}{\rho_{i}} = \underset{\alpha}{\sum} \pd{\Phi}{n_{\alpha}}\pd{n_{\alpha}}{\rho_{i}}
\end{equation}

% \begin{equation}
%   \frac{\mu^\RF_{i,\bulk}}{\kB T} = \int \biggl(\frac{Z-1}{\rho_i} \biggr) d\rho_i + Z - 1
% \end{equation}

% \begin{align}
%   \pd{\Phi^\RF_\bulk}{\rho_i} =& - \ln \lb 1 - n_3 \rb + \frac{n_0}{ \lb 1 - n_3 \rb}\pd{n_3}{\rho_i} \nonumber \\ &+
%   \frac{n_2}{\lb 1 - n_3 \rb}\pd{n_1}{\rho_i} + \frac{n_1}{\lb 1 - n_3 \rb}\pd{n_2}{\rho_i} + \frac{n_1 n_2}{\lb 1 - n_3 \rb^2}\pd{n_3}{\rho_i} \nonumber \\ &+
%   \frac{3 n_2^2 }{24\pi \lb 1 - n_3 \rb^2}\pd{n_2}{\rho_i} + \frac{2 n_2^3 }{24\pi \lb 1 - n_3 \rb^3}\pd{n_3}{\rho_i}
% \end{align}
For the bulk limit we have
\begin{align}
  n_{0,\bulk} &= \overset{\NC}{\underset{i=1}{\sum}} \rho_{i,\bulk}\\
  n_{1,\bulk} &= \overset{\NC}{\underset{i=1}{\sum}} R_i \rho_{i,\bulk}\\
  n_{2,\bulk} &= 4 \pi \overset{\NC}{\underset{i=1}{\sum}} R_i^2 \rho_{i,\bulk}\\
  n_{3,\bulk} &= \frac{4 \pi}{3} \overset{\NC}{\underset{i=1}{\sum}} R_i^3 \rho_{i,\bulk}
\end{align}
and
\begin{align}
  \pd{n_{0,\bulk}}{\rho_{i,\bulk}} &= 1\\
  \pd{n_{1,\bulk}}{\rho_{i,\bulk}} &=  R_i\\
  \pd{n_{2,\bulk}}{\rho_{i,\bulk}} &= 4 \pi R_i^2\\
  \pd{n_{3,\bulk}}{\rho_{i,\bulk}} &= \frac{4 \pi}{3} R_i^3
\end{align}
leading to
\begin{equation}
  \beta \mu^i_{\excess,\bulk} = \pd{\Phi}{n_{0,\bulk}} + R_i \pd{\Phi}{n_{1,\bulk}} + 4 \pi R_i^2 \pd{\Phi}{n_{2,\bulk}} + \frac{4 \pi R_i^3}{3} \pd{\Phi}{n_{3,\bulk}}
\end{equation}

\subsection{The Rosenfeld functional}
In the bulk phase (delete vector weight contributions) the Rosenfeld functional reduces to
\begin{equation}
  \Phi^\RF_\bulk = -n_0 \ln \lb 1 - n_3 \rb +
  \frac{n_1 n_2}{1 - n_3} +
  \frac{n_2^3 }{24\pi \lb 1 - n_3 \rb^2}
\end{equation}
which is the scaled particle theory (SPT) Helmholtz energy equation
for mixtures. The SPT EOS is identical to the Percus–Yevick EOS.

The bulk differentials become,
\begin{align}
  \pd{\Phi^\RF}{n_{0,\bulk}} &= -\ln \lb 1 - n_{3,\bulk} \rb \\
  \pd{\Phi^\RF}{n_{1,\bulk}} &= \frac{n_{2,\bulk}}{1 - n_{3,\bulk}} \\
  \pd{\Phi^\RF}{n_{2,\bulk}} &= \frac{n_{1,\bulk}}{1 - n_{3,\bulk}} + \frac{n_{2,\bulk}^2}{8\pi \lb 1 - n_{3,\bulk} \rb^2} \\
  \pd{\Phi^\RF}{n_{3,\bulk}} &= \frac{n_{0,\bulk}}{1 - n_{3,\bulk}} +
  \frac{n_{1,\bulk} n_{2,\bulk}}{\lb 1 - n_{3,\bulk} \rb^2} +
  \frac{n_{2,\bulk}^3}{12\pi \lb 1 - n_{3,\bulk} \rb^3}
\end{align}

\begin{align}
  \beta p_{\excess} + \Phi =& \underset{i=1}{\sum} \pd{\Phi}{n_{\alpha}}n_{\alpha} \nonumber \\
  = & -n_{0,\bulk} \ln \lb 1 - n_{3,\bulk} \rb \nonumber \\
      &+ n_{1,\bulk}\frac{n_{2,\bulk}}{1 - n_{3,\bulk}} \nonumber \\
      &+ n_{2,\bulk} \biggl(\frac{n_{1,\bulk}}{1 - n_{3,\bulk}} + \frac{n_{2,\bulk}^2}{8\pi \lb 1 - n_{3,\bulk} \rb^2} \biggr) \nonumber \\
      &+ n_{3,\bulk} \biggl(\frac{n_{0,\bulk}}{1 - n_{3,\bulk}} + 
      \frac{n_{1,\bulk} n_{2,\bulk}}{\lb 1 - n_{3,\bulk} \rb^2} +
        \frac{n_{2,\bulk}^3}{12\pi \lb 1 - n_{3,\bulk} \rb^3} \biggr) \nonumber\\
    = & -n_{0,\bulk} \ln \lb 1 - n_{3,\bulk} \rb 
      + \frac{2n_{1,\bulk}n_{2,\bulk}}{1 - n_{3,\bulk}}
      + \frac{n_{2,\bulk}^3}{8\pi \lb 1 - n_{3,\bulk} \rb^2}  \nonumber \\
      &+ n_{3,\bulk} \biggl(\frac{n_{0,\bulk}}{1 - n_{3,\bulk}} + 
      \frac{n_{1,\bulk} n_{2,\bulk}}{\lb 1 - n_{3,\bulk} \rb^2} +
        \frac{n_{2,\bulk}^3}{12\pi \lb 1 - n_{3,\bulk} \rb^3} \biggr) \\
  \beta p_{\excess} &= \frac{n_{1,\bulk}n_{2,\bulk}}{1 - n_{3,\bulk}}
      + \frac{n_{2,\bulk}^3}{8\pi \lb 1 - n_{3,\bulk} \rb^2}  \nonumber \\
      &+ n_{3,\bulk} \biggl(\frac{n_{0,\bulk}}{1 - n_{3,\bulk}} + 
      \frac{n_{1,\bulk} n_{2,\bulk}}{\lb 1 - n_{3,\bulk} \rb^2} +
        \frac{n_{2,\bulk}^3}{12\pi \lb 1 - n_{3,\bulk} \rb^3} \biggr) \nonumber\\
                            &- \frac{n_{2,\bulk}^3 }{24\pi \lb 1 - n_{3,\bulk} \rb^2} \nonumber\\
                            &= \frac{n_{0,\bulk}n_{3,\bulk}}{\lb 1 - n_{3,\bulk} \rb} + \frac{n_{1,\bulk} n_{2,\bulk}}{\lb 1 - n_{3,\bulk} \rb^2} + \frac{n_{2,\bulk}^3}{12 \pi\lb 1 - n_{3,\bulk} \rb^3}
\end{align}
Adding the ideal contribution, $\beta p_{\ideal} = n_{0,\bulk}$, and
dividing by $n_{0,\bulk}$ we get the compressibillity of the SPT EOS,
\begin{equation}
  z^\RF_\bulk = \frac{p}{n_0 \kB T} = \frac{1}{\lb 1 - n_3 \rb} + \frac{n_1 n_2}{n_0} \frac{1}{\lb 1 - n_3 \rb^2} + \frac{n_2^3}{12 \pi n_0} \frac{1}{\lb 1 - n_3 \rb^3},
\end{equation}
and for a singel component the equation reduces to
\begin{equation}
  z^\RF_{\bulk,\pure} = \frac{1 + n_3 + n_3^2}{\lb 1 - n_3 \rb^3}.
\end{equation}

For the pure fluid, using $\eta = n_{3,\bulk}$ and \eqref{eq:chem_pot} we get,
\begin{align}
  \hat{\mu}^\pure_{\excess,\bulk} =& -\ln \lb 1 - \eta \rb + \frac{3 \eta}{1 - \eta} + \frac{3 \eta}{1 - \eta} + \frac{36\pi\eta^2}{8\pi \lb 1 - \eta \rb^2}\nonumber \\
  & +\frac{\eta}{1 - \eta} +
  \frac{3 \eta^2}{\lb 1 - \eta \rb^2} +
    \frac{36\pi\eta^3}{12\pi \lb 1 - \eta \rb^3}\nonumber \\
   =&-\ln \lb 1 - \eta \rb + \frac{7 \eta}{1 - \eta} + \frac{15\eta^2}{2\lb 1 - \eta \rb^2}+
      \frac{3\eta^3}{\lb 1 - \eta \rb^3}\nonumber \\
  =&\frac{14\eta - 13\eta^2 + 5\eta^3}{2\lb 1 - \eta \rb^3}-\ln \lb 1 - \eta \rb
\end{align}


\clearpage
\bibliographystyle{plainnat}
\bibliography{./DFT.bib,./HardSphere.bib,./PC-SAFT.bib}

\end{document}

%%% Local Variables: 
%%% mode: latex
%%% TeX-master: t
%%% ispell-local-dictionary: "british"
%%% End: 
