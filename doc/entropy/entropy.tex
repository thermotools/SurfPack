
\documentclass[12pt, letterpaper]{article}
\pdfminorversion=4
\usepackage[utf8]{inputenc}
\usepackage{textcomp}
\usepackage{listings}
\usepackage{babel}
\usepackage[square,numbers]{natbib}
%\usepackage{amsmath}
\usepackage{amssymb,amsmath,amsthm,amsfonts}
\usepackage{pslatex}
%\usepackage{bm}% better \boldsymbol
\usepackage{array}% improves tabular environment.
\usepackage{dcolumn}% also improves tabular environment, with decimal centring.
%\usepackage{lastpage}
\usepackage{listings}
\usepackage{tikz}
\usetikzlibrary{decorations.pathmorphing}
%\usepackage{pgfplots}
%\usepackage{cleveref}
\usepackage{pgf,pgfarrows,pgfnodes}

\usepackage[final]{pdfpages}
\usepackage{hyperref}
\usepackage{doi}

%\usepackage[altbullet,expert]{lucidabr}
%\usepackage[charter]{mathdesign}
\usepackage{cleveref}
\usepackage{ifthen}
%\usepackage{nomencl}
\usepackage{parskip}
\usepackage{psfrag}
\usepackage{booktabs}
\usepackage{pifont}
%\usepackage{subfig,caption}
\usepackage{microtype}% for pdflatex; medisin mot overfull \vbox
\usepackage{xspace}% for pdflatex; medisin mot overfull \vbox
\usepackage[font=footnotesize]{subcaption}
\usepackage{siunitx}
\usepackage{todonotes}
\presetkeys%
    {todonotes}%
    {inline,backgroundcolor=orange}{}

%% Formattering av figur- og tabelltekster
%
%
% Egendefinerte
%
%\newcommand*{\od}[3][]{\frac{\mathrm{d}^{#1}#2}{\mathrm{d}{#3}^{#1}}}% ordinary derivative
\newcommand*{\od}[3][]{\frac{\dif^{#1}#2}{\dif{#3}^{#1}}}% ordinary derivative
\newcommand*{\pd}[3][]{\frac{\partial^{#1}#2}{\partial{#3}^{#1}}}% partial derivative
\newcommand*{\pdt}[3][]{{\partial^{#1}#2}/{\partial{#3}^{#1}}}% partial
                                % derivative for inline use.
\newcommand{\pone}[3]{\frac{\partial #1}{\partial #2}_{#3}}% partial
                                % derivative with information of
                                % constant variables
\newcommand{\ponel}[3]{\frac{\partial #1}{\partial #2}\bigg|_{#3}} % partial derivative with informatio of constant variable. A line is added.
\newcommand{\ptwo}[3]{\frac{\partial^{2} #1}{\partial #2 \partial
    #3}} % partial differential in two different variables
\newcommand{\pdn}[3]{\frac{\partial^{#1}#2}{\partial{#3}^{#1}}}% partial derivative

% Total derivative:
\newcommand*{\ttd}[2]{\frac{\mathrm{D} #1}{\mathrm{D} #2}}
\newcommand*{\td}[2]{\frac{\mathrm{d} #1}{\mathrm{d} #2}}
\newcommand*{\ddt}{\frac{\partial}{\partial t}}
\newcommand*{\ddx}{\frac{\partial}{\partial x}}
% Vectors etc:
% For Computer Modern:

\DeclareMathAlphabet{\mathsfsl}{OT1}{cmss}{m}{sl}
%\renewcommand*{\vec}[1]{\boldsymbol{#1}}%
\newcommand*{\vc}[1]{\vec{\mathbf{#1}}}%
\newcommand*{\tensor}[1]{\mathsfsl{#1}}% 2. order tensor
\newcommand*{\matr}[1]{\tensor{#1}}% matrix
\renewcommand*{\div}{\boldsymbol{\nabla\cdot}}% divergence
\newcommand*{\grad}{\boldsymbol{\nabla}}% gradient
% fancy differential from Claudio Beccari, TUGboat:
% adjusts spacing automatically
\makeatletter
\newcommand*{\dif}{\@ifnextchar^{\DIfF}{\DIfF^{}}}
\def\DIfF^#1{\mathop{\mathrm{\mathstrut d}}\nolimits^{#1}\gobblesp@ce}
\def\gobblesp@ce{\futurelet\diffarg\opsp@ce}
\def\opsp@ce{%
  \let\DiffSpace\!%
  \ifx\diffarg(%
    \let\DiffSpace\relax
  \else
    \ifx\diffarg[%
      \let\DiffSpace\relax
    \else
      \ifx\diffarg\{%
        \let\DiffSpace\relax
      \fi\fi\fi\DiffSpace}
\makeatother
%
\newcommand*{\me}{\mathrm{e}}% e is not a variable (2.718281828...)
%\newcommand*{\mi}{\mathrm{i}}%  nor i (\sqrt{-1})
\newcommand*{\mpi}{\uppi}% nor pi (3.141592...) (works for for Lucida)
%
% lav tekst-indeks/subscript/pedex
\newcommand*{\ped}[1]{\ensuremath{_{\text{#1}}}}
% hy tekst-indeks/superscript/apex
\newcommand*{\ap}[1]{\ensuremath{^{\text{#1}}}}
\newcommand*{\apr}[1]{\ensuremath{^{\mathrm{#1}}}}
\newcommand*{\pedr}[1]{\ensuremath{_{\mathrm{#1}}}}
%
\newcommand*{\volfrac}{\alpha}% volume fraction
\newcommand*{\surften}{\sigma}% coeff. of surface tension
\newcommand*{\curv}{\kappa}% curvature
\newcommand*{\ls}{\phi}% level-set function
\newcommand*{\ep}{\Phi}% electric potential
\newcommand*{\perm}{\varepsilon}% electric permittivity
\newcommand*{\visc}{\mu}% molecular (dymamic) viscosity
\newcommand*{\kvisc}{\nu}% kinematic viscosity
\newcommand*{\cfl}{C}% CFL number

% Grid
\newcommand{\jj}{j}
\newcommand{\jph}{{j+1/2}}
\newcommand{\jmh}{{j-1/2}}
\newcommand{\jp}{{j+1}}
\newcommand{\jm}{{j-1}}
\newcommand{\nn}{n}
\newcommand{\nph}{{n+1/2}}
\newcommand{\nmh}{{n-1/2}}
\newcommand{\np}{{n+1}}
%
\newcommand{\lf}{\text{LF}}
\newcommand{\lw}{\text{Ri}}
%
\newcommand*{\cons}{\vec U}
\newcommand*{\flux}{\vec F}
\newcommand*{\dens}{\rho}
\newcommand*{\svol}{\ensuremath v}
\newcommand*{\temp}{\ensuremath T}
\newcommand*{\vel}{\ensuremath u}
\newcommand*{\mom}{\dens\vel}
\newcommand*{\toten}{\ensuremath E}
\newcommand*{\inten}{\ensuremath e}
\newcommand*{\press}{\ensuremath p}
\renewcommand*{\ss}{\ensuremath a}
\newcommand*{\jac}{\matr A}
%
\newcommand*{\abs}[1]{\lvert#1\rvert}
\newcommand*{\bigabs}[1]{\bigl\lvert#1\bigr\rvert}
\newcommand*{\biggabs}[1]{\biggl\lvert#1\biggr\rvert}
\newcommand*{\norm}[1]{\lVert#1\rVert}
%
\newcommand*{\e}[1]{\times 10^{#1}}
\newcommand*{\ex}[1]{\times 10^{#1}}%shorthand -- for use e.g. in tables
\newcommand*{\exi}[1]{10^{#1}}%shorthand -- for use e.g. in tables
\newcommand*{\nondim}[1]{\ensuremath{\mathit{#1}}}% italic iflg. ISO. (???)
\newcommand*{\rey}{\nondim{Re}}
\newcommand*{\acro}[1]{\textsc{\MakeLowercase{#1}}}%acronyms etc.

\newcommand{\nto}{\ensuremath{\mbox{N}_{\mbox{\scriptsize 2}}}}
\newcommand{\chfire}{\ensuremath{\mbox{CH}_{\mbox{\scriptsize 4}}}}
%\newcommand*{\checked}{\ding{51}}
\newcommand{\coto}{\ensuremath{\text{CO}_{\text{\scriptsize 2}}}}
\newcommand{\celsius}{\ensuremath{^\circ\text{C}}}
%\newcommand{\clap}{Clapeyron~}
\newcommand{\subl}{\ensuremath{\text{sub}}}
\newcommand{\spec}{\text{spec}}
\newcommand{\sat}{\text{sat}}
\newcommand{\sol}{\text{sol}}
\newcommand{\liq}{\text{liq}}
\newcommand{\vap}{\text{vap}}
\newcommand{\amb}{\text{amb}}
\newcommand{\tr}{\text{tr}}
\newcommand{\crit}{\text{crit}}
\newcommand{\entr}{\ensuremath{\text{s}}}
\newcommand{\fus}{\text{fus}}
\newcommand{\flash}[1]{\ensuremath{#1\text{-flash}}}
\newcommand{\spce}[2]{\ensuremath{#1\, #2\text{ space}}}
\newcommand{\spanwagner}{\text{Span--Wagner}}
\newcommand{\triplepoint}{\text{TP triple point}}
\newcommand{\wrpt}{\text{with respec to~}}
%\sisetup{input-symbols = {( )}}
% \newcommand*{\red}{\ensuremath{\text{R}}\xspace}
% \newcommand*{\crit}{\ensuremath{\text{c}}\xspace}
% \newcommand*{\mix}{\ensuremath{\text{m}}\xspace}
\newcommand*{\lb}{\ensuremath{\left(}}
\newcommand*{\rb}{\ensuremath{\right)}}
\newcommand*{\lbf}{\ensuremath{\left[}}
\newcommand*{\rbf}{\ensuremath{\right]}}
\newcommand{\LEFT}{\ensuremath{\text{L}}\xspace}
\newcommand{\RIGTH}{\ensuremath{\text{R}}\xspace}
\newcommand{\cdft}{\ensuremath{\text{classical DFT}}\xspace}
\newcommand{\RF}{\ensuremath{\text{RF}}\xspace}
\newcommand{\WB}{\ensuremath{\text{WB}}\xspace}
\newcommand{\WBII}{\ensuremath{\text{WBII}}\xspace}
\newcommand{\excess}{\ensuremath{\text{ex}}\xspace}
\newcommand{\ideal}{\ensuremath{\text{id}}\xspace}
\newcommand{\rvec}{\ensuremath{\mathbf{r}}\xspace}
\newcommand{\pvec}{\ensuremath{\mathbf{p}}\xspace}
\newcommand{\attractive}{\ensuremath{\text{att}}\xspace}
\newcommand{\kB}{\ensuremath{\text{k}_{\text{B}}}\xspace}
\newcommand{\NA}{\ensuremath{N_{\text{A}}}\xspace}
\newcommand{\external}{\ensuremath{\text{ext}}\xspace}
\newcommand{\bulk}{\ensuremath{\text{b}}\xspace}
\newcommand{\pure}{\ensuremath{\text{p}}\xspace}
\newcommand{\NC}{\ensuremath{\text{NC}}\xspace}
\newcommand{\disp}{\ensuremath{\text{disp}}\xspace}
\graphicspath{{gfx/}}

\title{Classical Density Functional Theory for Thermopack - Entropy}
\author{Morten Hammer}
\date{\today}

\begin{document}
%\tableofcontents
%\printnomenclature


\begin{titlepage}
\maketitle
\end{titlepage}

\section{Introduction}

The entropy ($\tilde{S}$ (\si{\joule\per\kelvin})) is generally given
from the Helmholtz energy ($\tilde{F}$ (\si{\joule})) differential
with respect to temperature,

\begin{equation}
  \tilde{S} = - \od{\tilde{F}}{T}\bigg|_{V,n}.
\end{equation}

The Helmholtz functional in DFT is given as
\begin{equation}
  F\left[\rho\right] = \frac{\tilde{F}}{V}.
\end{equation}

The mol specific Helmholtz functional is then given as
\begin{equation}
  f\left[\rho\right] = \frac{F}{\rho}.
\end{equation}

The mol specific entropy then becomes
\begin{equation}
  s = - \od{f}{T}\bigg|_{\rho} = - \frac{1}{\rho}\od{F}{T}\bigg|_{\rho}.
\end{equation}

This is different from the definition of \citet{stierle2021}, where mol specific entropy is defined as
\begin{equation}
  s = - \frac{1}{\bar{\rho}}\od{F}{T}\bigg|_{\rho},
\end{equation}
where $\bar{\rho}$ is a weighted density. $\bar{\rho}$ is not constant upon differentiating with temperature.


The functional depends on the density through the weighted densities,
\begin{equation}
  \label{eq:n_alpha}
  n_\alpha = \int d \mathbf{r}^\prime \rho \lb \mathbf{r}^\prime \rb w_\alpha \lb \mathbf{r} - \mathbf{r}^\prime \rb =  \rho \lb \mathbf{r} \rb \otimes w_\alpha \lb \mathbf{r}\rb,
\end{equation}
where $w_\alpha$ is the a weight function. In effect the the functional becomes,

\begin{equation}
  F = F\lb \mathbf{n}\lb T\rb, T\rb
\end{equation}

The temperature differential of $F$ then becomes,
\begin{equation}
  \od{F}{T} = \underset{\alpha}{\sum} \pd{F}{n_\alpha}\pd{n_\alpha}{T} + \pd{F}{T}
\end{equation}

Differentiating  the equation for $n_\alpha$, \cref{eq:n_alpha}, give,
\begin{align}
  \label{eq:n_alpha_diff}
  \pd{n_\alpha}{T} &= \pd{}{T} \rho \lb \mathbf{r} \rb \otimes w_\alpha \lb \mathbf{r}\rb \nonumber \\
                   &= \pd{}{T} \mathcal{F}^{-1} \left[ \mathcal{F}\left[\rho \lb \mathbf{r} \rb\right] \mathcal{F}\left[w_\alpha \lb \mathbf{r}\rb\right]  \right] \nonumber \\
                   &= \pd{}{T} \mathcal{F}^{-1} \left[ \hat{\rho} \lb \mathbf{k} \rb \hat{w}_\alpha \lb \mathbf{k}\rb  \right] \nonumber \\
                   &= \mathcal{F}^{-1} \left[ \hat{\rho} \lb \mathbf{k} \rb \pd{\hat{w}_\alpha \lb \mathbf{k}\rb}{T}  \right]
\end{align}

The weight functions in Fourier space is given in the supplementary information by \citet{stierle2020a}.

\begin{align}
  \label{eq:weight_functions_fourier}
  \hat{w}_0 \lb \mathbf{k}\rb &= j_0\lb 2 \pi R |\mathbf{k}| \rb  \\
  \hat{w}_1 \lb \mathbf{k}\rb &= R j_0\lb 2 \pi R |\mathbf{k}| \rb  \\
  \hat{w}_2 \lb \mathbf{k}\rb &= 4 \pi R^2 j_0\lb 2 \pi R |\mathbf{k}| \rb  \\
  \hat{w}_3 \lb \mathbf{k}\rb &= \frac{4}{3} \pi R^3 \lb j_0\lb 2 \pi R |\mathbf{k}| \rb + j_2\lb 2 \pi R |\mathbf{k}| \rb \rb  \\
  \hat{w}_{V1} \lb \mathbf{k}\rb &= - \frac{i \mathbf{k}}{2 R} \hat{w}_3 \lb \mathbf{k}\rb  \\
  \hat{w}_{V2} \lb \mathbf{k}\rb &= - 2 \pi i \mathbf{k} \hat{w}_3 \lb \mathbf{k}\rb  \\
  \hat{w}_{\disp} \lb \mathbf{k}\rb &= j_0\lb 4 \pi \psi R |\mathbf{k}| \rb + j_2\lb 4 \pi \psi R |\mathbf{k}| \rb
\end{align}
Here $j$ is the spherical Bessel functions of the first kind of order
zero and two. The Bessel functions of order 0-2 is given as
\begin{align}
  \label{eq:bessel_functions}
  j_0\lb \xi \rb &= \frac{\sin \lb \xi \rb}{\xi} \\
  j_1\lb \xi \rb &= \frac{\sin \lb \xi \rb}{\xi^2} - \frac{\cos \lb \xi \rb}{\xi}  \\
  j_2\lb \xi \rb &= \biggl(\frac{3}{\xi^2} - 1\biggr) \frac{\sin \lb \xi \rb}{\xi} - \frac{3 \cos \lb \xi \rb}{\xi^2}.
\end{align}
Differentiating the Bessel functions of order 0 and three other useful relations,
\begin{align}
  \label{eq:bessel_function_differentials}
  \pd{j_0\lb \xi \rb}{\xi} &= -\frac{\sin \lb \xi \rb}{\xi^2} + \frac{\cos \lb \xi \rb}{\xi} = - j_1\lb \xi \rb\\
  \pd{\left[\xi^2\lb j_0\lb \xi \rb + j_2\lb \xi \rb\rb\right]}{\xi} &= \pd{}{\xi} \biggl(\frac{3\sin \lb \xi \rb}{\xi} - 3 \cos \lb \xi \rb\biggr) \nonumber \\
                           &= -\frac{3\sin \lb \xi \rb}{\xi^2} + \frac{3\cos \lb \xi \rb}{\xi} + 3 \sin \lb \xi \rb \nonumber \\
                           &= -3j_1\lb \xi \rb + 3 \xi j_0\lb \xi \rb\\
  \pd{\left[\xi^3\lb j_0\lb \xi \rb + j_2\lb \xi \rb\rb\right]}{\xi} &= 3 \xi^2 j_0\lb \xi \rb \\
  \pd{\lb j_0\lb \xi \rb + j_2\lb \xi \rb\rb}{\xi} &= 3\pd{}{\xi} \biggl(\frac{\sin \lb \xi \rb}{\xi^3} - \frac{\cos \lb \xi \rb}{\xi^2}\biggr) \nonumber \\
                           &= 3 \biggl(3\frac{\cos \lb \xi \rb}{\xi^3} - 3 \frac{\sin \lb \xi \rb}{\xi^4} + \frac{\sin \lb \xi \rb}{\xi^2}\biggr) \nonumber \\
                           &= -\frac{3}{\xi} j_2\lb \xi \rb
\end{align}

The differentials of the weight functions in Fourier space then becomes,
\begin{align}
  \label{eq:weight_functions_fourier_diff}
  \pd{\hat{w}_0 \lb \mathbf{k}\rb}{R} &= - 2 \pi |\mathbf{k}| j_1\lb 2 \pi R |\mathbf{k}| \rb  \\
  \pd{\hat{w}_1 \lb \mathbf{k}\rb}{R} &= j_0\lb 2 \pi R |\mathbf{k}| \rb - 2 \pi R |\mathbf{k}| j_1\lb 2 \pi R |\mathbf{k}| \rb \\
  \pd{\hat{w}_2 \lb \mathbf{k}\rb}{R} &= 8 \pi R j_0\lb 2 \pi R |\mathbf{k}| \rb - 8 \pi^2 R^2 |\mathbf{k}| j_1\lb 2 \pi R |\mathbf{k}| \rb  \\
  \pd{\hat{w}_3 \lb \mathbf{k}\rb}{R}  &=  \frac{1}{6 \pi^2 |\mathbf{k}|^3} \pd{}{R}\biggl( \lb 2 \pi R |\mathbf{k}|\rb ^3 \lb j_0\lb 2 \pi R |\mathbf{k}| \rb + j_2\lb 2 \pi R |\mathbf{k}| \rb \rb \biggr)  \nonumber \\
                                      &=  \frac{1}{6 \pi^2 |\mathbf{k}|^3} 3 \lb 2 \pi R |\mathbf{k}|\rb ^2  j_0\lb 2 \pi R |\mathbf{k}| \rb 2 \pi |\mathbf{k}| \nonumber  \\
                                      &=  4 \pi R^2 j_0\lb 2 \pi R |\mathbf{k}| \rb  \\
  \pd{\hat{w}_{V1} \lb \mathbf{k}\rb}{R} &= - \frac{2 \pi i \mathbf{k}}{3 (2 \pi |\mathbf{k}|)^2}  \pd{}{R}\biggl( (2 \pi R |\mathbf{k}|)^2 \lb j_0\lb 2 \pi R |\mathbf{k}| \rb + j_2\lb 2 \pi R |\mathbf{k}| \rb \rb \biggr) \nonumber \\
                                      &= - \frac{2 \pi i \mathbf{k}}{3 (2 \pi |\mathbf{k}|)^2} \biggl( 6 \pi R |\mathbf{k}| j_0\lb 2 \pi R |\mathbf{k}| \rb - 3 j_1\lb 2 \pi R |\mathbf{k}| \rb \biggr) 2 \pi |\mathbf{k}|  \nonumber \\
   &= - i \biggl( 2 \pi R \mathbf{k} j_0\lb 2 \pi R |\mathbf{k}| \rb - \frac{\mathbf{k}}{|\mathbf{k}|} j_1\lb 2 \pi R |\mathbf{k}| \rb \biggr)  \\
  \pd{\hat{w}_{V2} \lb \mathbf{k}\rb}{R} &= - 8 \pi^2 i \mathbf{k} R^2 j_0\lb 2 \pi R |\mathbf{k}| \rb  \\
  \pd{\hat{w}_{\disp} \lb \mathbf{k}\rb}{R} &= -\frac{3 j_2\lb 4 \pi \psi R |\mathbf{k}| \rb}{4 \pi \psi R |\mathbf{k}|} 4 \pi \psi |\mathbf{k}| \nonumber \\
                                      &= -\frac{3 j_2\lb 4 \pi \psi R |\mathbf{k}| \rb}{R}
\end{align}

Finally,
\begin{equation}
  \pd{\hat{w}_\alpha \lb \mathbf{k}\rb}{T} = \pd{\hat{w}_\alpha \lb \mathbf{k}\rb}{R} \pd{R}{T}
\end{equation}

\subsection{Density shift contributions}

When shifting the density profiles we get contributions from convolution of constant densities ($\rho^{\rm{s}}$). These contributions are given as,
\begin{align}
  \label{eq:n_alpha_diff_steady}
  n_\alpha^{\rm{s}} &= \rho^{\rm{s}} \lb \mathbf{r} \rb \otimes w_\alpha \lb \mathbf{r}\rb = \rho^{\rm{s}} w_\alpha^{\rm{s}} \\
  \pd{n_\alpha^{\rm{s}}}{T} &= \rho^{\rm{s}} \pd{w_\alpha^{\rm{s}}}{T}.
\end{align}
Where,
\begin{align}
  \label{eq:weight_functions_steady}
  w_0^{\rm{s}} &= 1  \\
  w_1^{\rm{s}} &= R  \\
  w_2^{\rm{s}} &= 4 \pi R^2  \\
  w_3^{\rm{s}} &= \frac{4}{3} \pi R^3  \\
  w_{V1}^{\rm{s}} &= 0  \\
  w_{V2}^{\rm{s}} &= 0  \\
  w_{\disp}^{\rm{s}} &= 1.
\end{align}

Differentiating we get,
\begin{align}
  \label{eq:weight_functions_steady_diff}
  \pd{w_0^{\rm{s}}}{R} &= 0  \\
  \pd{w_1^{\rm{s}}}{R} &= 1  \\
  \pd{w_2^{\rm{s}}}{R} &= 8 \pi R  \\
  \pd{w_3^{\rm{s}}}{R} &= 4 \pi R^2  \\
  \pd{w_{V1}^{\rm{s}}}{R} &= 0  \\
  \pd{w_{V2}^{\rm{s}}}{R} &= 0  \\
  \pd{w_{disp}^{\rm{s}}}{R} &= 0
\end{align}


\subsection{Generic weight functions}
Generic weight functions of Dirac delta and heaviside functions are,
\begin{align}
  \label{eq:generic_weight_functions_fourier}
  \hat{w}_\delta \lb \mathbf{k}\rb &=  \beta(R) \tilde{w}_\delta = \beta(R) 4 \pi (R \Psi)^2 j_0\lb 2 \pi R \Psi |\mathbf{k}| \rb \\
  \hat{w}_\Theta \lb \mathbf{k}\rb &=  \beta(R) \tilde{w}_\Theta = \beta(R) \frac{4}{3} \pi (R\Psi)^3 \lb j_0\lb 2 \pi R\Psi |\mathbf{k}| \rb + j_2\lb 2 \pi R\Psi |\mathbf{k}| \rb \rb  \\
\end{align}
Here, $\Psi$ simply scale the width of the weight, $\beta(R)$ is a
prefactor depending on $R$. We also introduce the utility variable
$\lambda(R)$,
\begin{align}
  \label{eq:lambdas}
  \lambda_\delta \lb R \rb &=  \beta(R) 4 \pi (R \Psi)^2  \\
  \lambda_\Theta  \lb R \rb &=  \beta(R) \tilde{w}_\Theta = \beta(R) \frac{4}{3} \pi (R\Psi)^3.
\end{align}

The differentials of the generic weight functions in Fourier space then becomes,
\begin{align}
  \label{eq:generic_weight_functions_fourier_diff}
  \pd{\hat{w}_\delta \lb \mathbf{k}\rb}{R} &=  \lambda_{\delta,R} j_0\lb 2 \pi R \Psi |\mathbf{k}| \rb - \lambda(R) 2 \pi R \Psi |\mathbf{k}| j_1\lb 2 \pi R \Psi |\mathbf{k}| \rb  \\
  \pd{\hat{w}_\Theta \lb \mathbf{k}\rb}{R}  &=  \frac{\beta(R)}{6 \pi^2 |\mathbf{k}|^3} \pd{}{R}\biggl( \lb 2 \pi R \Psi |\mathbf{k}|\rb ^3 \lb j_0\lb 2 \pi R \Psi |\mathbf{k}| \rb + j_2\lb 2 \pi R \Psi |\mathbf{k}| \rb \rb \biggr) + \beta_R \tilde{w}_\Theta  \nonumber \\
                                           &=  \frac{\beta(R)}{6 \pi^2 |\mathbf{k}|^3} 3 \lb 2 \pi R \Psi |\mathbf{k}|\rb ^2  j_0\lb 2 \pi R \Psi |\mathbf{k}| \rb 2 \pi \Psi |\mathbf{k}| + \beta_R \tilde{w}_\Theta \nonumber  \\
                                           &=  \beta(R) 4 \pi R^2 \Psi^2 j_0\lb 2 \pi R \Psi |\mathbf{k}| \rb + \beta_R \tilde{w}_\Theta
\end{align}


\subsubsection{Generic density shift contributions}
Where,
\begin{align}
  \label{eq:generic_weight_functions_steady}
  w_\delta^{\rm{s}} &= \beta(R) 4 \pi (R \Psi)^2 = \lambda_\delta(R)  \\
  w_\Theta^{\rm{s}} &= \beta(R) \frac{4}{3} \pi (R \Psi)^3 = \lambda_\Theta(R).
\end{align}

Differentiating we get,
\begin{align}
  \label{eq:generic_weight_functions_steady_diff}
  \pd{w_\delta^{\rm{s}}}{R} &= \lambda_{\delta,R} \\
  \pd{w_\Theta^{\rm{s}}}{R} &= \lambda_{\Theta,R}
\end{align}


\section{Other excess quantities}
Knowing the entropy all the other excess energy densities can be
calculated. The internal energy density is known from
\begin{equation}
  \label{eq:energy}
  u^\excess = f^\excess + T s^\excess.
\end{equation}
Further, the enthalpy is known from,
\begin{equation}
  \label{eq:enthalpy}
  h^\excess = T s^\excess + \underset{i}{\sum} \rho_i \mu_i^\excess,
\end{equation}
where the excess chemical potential can be calculated from
\begin{equation}
  \label{eq:chem_pot}
  \mu_i^\excess = \mu_i - \kB T \ln \lb \rho_i \rb - \rm{Const.}
\end{equation}
as the chemical potential, $\mu_i$, is always specified for the
simulations. \todo{Effect of external potential?}
% The chemical potential is also known from
% \begin{equation}
%   \ref{eq:chem_pot_diff}
%   \mu_i^\excess = \underset{\alpha}{\sum} \pd{F}{n_\alpha} \pd{n_\alpha}{\rho_i}.
% \end{equation}

\section{Pressures}
For the planar geometry the parallel component of the pressure tensor
is given from the grand potential density,
\begin{equation}
  \label{eq:parallel_pressure}
  p_\parallel = - f + \underset{i}{\sum} \rho_i \mu_i.
\end{equation}


\section{Results}



\clearpage
\bibliographystyle{plainnat}
\bibliography{../bib/DFT.bib,../bib/HardSphere.bib,../bib/PC-SAFT.bib}

\end{document}

%%% Local Variables: 
%%% mode: latex
%%% TeX-master: t
%%% ispell-local-dictionary: "british"
%%% End: 
